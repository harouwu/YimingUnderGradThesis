% vim:ts=4:sw=4
%
% Copyright (c) 2008-2009 solvethis
% Copyright (c) 2010-2015 Casper Ti. Vector
% Public domain.
%
% 使用前请先仔细阅读 pkuthss 和 biblatex-caspervector 的文档,
% 特别是其中的 FAQ 部分和用红色强调的部分。
% 两者可在终端/命令提示符中用
%   texdoc pkuthss
%   texdoc biblatex-caspervector
% 调出。

% 采用了自定义的(包括大小写不同于原文件的)字体文件名,
% 并改动 ctex.cfg 等配置文件的用户请自行加入 nofonts 选项;
% 其它用户不用加入 nofonts 选项,加入之后反而会产生错误。
%
% 图书馆要求电子版论文的目录必须为黑色,
% 且某些教务要求打印版论文的文字部分为纯黑色而非灰度打印,
% 【因此最终打印和提交论文前,请将“colorlinks”改为“nocolorlinks”。】
\documentclass[UTF8, colorlinks, uppermark, openany]{pkuthss}

% 使用 biblatex 排版参考文献,并规定其格式。
% 默认按照引用顺序排序(“sorting = none”),详见 biblatex-caspervector 的文档
% (因为是默认设置所以其实不用写,不过出于完备性的考虑仍然在这里列出)。
% 若需要按照英文文献在前,中文文献在后排序,请设置“sorting = ecnty”;
% 若需要按照中文文献在前,英文文献在后排序,请设置“sorting = centy”。
\usepackage[backend = biber, style = caspervector, utf8, sorting = none]{biblatex}
% 提供近似于学校所要求的 Times New Roman / Arial 的字体。
\usepackage[defaultsups]{newtxtext}
\usepackage{newtxmath}
% 产生 originauth.tex 里的 \square。
\usepackage{latexsym}
\usepackage{cancel}
\usepackage{amsmath}
\usepackage{url}
\usepackage{listings}
\usepackage{color}
\usepackage{graphicx}

% the following tells mathastext to use typewriter
\usepackage[T1]{fontenc}
\usepackage{mathastext}
\MTfamily{\ttdefault}\Mathastext 

\usepackage{bussproofs}
\usepackage[]{algorithm2e}
\usepackage{soul}
\usepackage{cancel}
\usepackage{multirow}

\newcommand{\code}[1]{\texttt{\footnotesize #1}}
\newcommand{\todo}[1]{{\bf \{TODO: {#1}\}}}

\def\modify#1#2#3{{\small\underline{\sf{#1}}:} {\color{red}{\small #2}}
{{\color{red}\mbox{$\Rightarrow$}}} {\color{blue}{#3}}}
%\renewcommand{\modify}[3]{{#3}}

\newcommand{\yxmodify}[2]{\modify{Yingfei}{#1}{#2}}

\newcommand\mymargin[1]{\marginpar{{\flushleft\textsc\footnotesize {#1}}}}
\newcommand\yxmargin[1]{\mymargin{YX:\;#1}}

\newcommand{\figref}[1]{Figure~\ref{#1}}
\newcommand{\eqnref}[1]{violation~\eqref{#1}}
\newcommand{\secref}[1]{Section~\ref{#1}}
\newcommand{\tblref}[1]{Table~\ref{#1}} 
\newcommand{\smalltitle}[1]{{\smallskip \noindent \bf  {#1}.\ }}
\newcommand{\smalltitlecolon}[1]{{\smallskip \noindent \bf  {#1}:\ }}


\newcommand{\yxmodifyok}[2]{#2}

\newcommand{\mycomment}[2]{{\small\color{magenta}\underline{\sf{#1}}:} {\color{magenta}{\small #2}}}
\newcommand{\yxcomment}[1]{\mycomment{Yingfei}{#1}}

% \renewcommand{\yxcomment}[1]{}
% \renewcommand{\yxmodify}[2]{#2}


\newcommand{\adag}{A-CFG\xspace}
\newcommand{\anadag}{an A-CFG\xspace}
\newcommand{\Anadag}{An A-CFG\xspace}



%\theoremstyle{definition}
\newtheorem{definition}{Definition}
\newtheorem{decision}{Requirement}
\newtheorem{law}{Requirement}
\lstset{
  basicstyle = \footnotesize\tt,
  }

% 按学校要求设定参考文献列表中的条目之内及之间的距离。
\setlength{\bibitemsep}{3bp}
% 对于 linespread 值的计算过程有兴趣的同学可以参考 pkuthss-extra.sty。
\renewcommand*{\bibfont}{\zihao{5}\linespread{1.27}\selectfont}

% 设定文档的基本信息。
\pkuthssinfo{
	cthesisname = {本科学士学位论文}, ethesisname = {Bachelor Thesis},
	ctitle = {一个双向变换C预处理器的实现与实验}, etitle = {Bidirectionalizing C Preprocessor},
	cauthor = {吴逸鸣},
	eauthor = {Yiming Wu},
	studentid = {1100012807},
	date = {2015.05.28},
	school = {信息科学技术学院},
	cmajor = {计算机科学与技术}, emajor = {Computer Sciense \& Engineering},
	direction = {计算机技术},
	cmentor = {熊英飞}, ementor = {Prof.\ Yingfei Xiong},
	ckeywords = {双向变换,预处理器,重写规则}, ekeywords = {Bidirectional Transformation, Preprocessor, Rewriting Rule}
}
% 载入参考文献数据库(注意不要省略“.bib”)。
%\addbibresource{thesis.bib}
\addbibresource{ASE15/main.bib}
\addbibresource{ASE15/reference.bib}

% 普通用户可删除此段。
\usepackage{color}
\def\pkuthssffaq{%
	\emph{\textcolor{red}{pkuthss 文档模版最常见问题:}}

	在最终打印和提交论文之前,
	请将 pkuthss 文档类选项中的 %
	\texttt{colorlinks} 改为 \texttt{nocolorlinks},
	因为图书馆要求电子版论文的目录必须为黑色,
	且某些教务要求打印版论文的文字部分为纯黑色而非灰度打印。

	\texttt{\string\cite}、\texttt{\string\parencite} %
	和 \texttt{\string\supercite} 三个命令分别产生%
	未格式化的、带方括号的和上标且带方括号的引用标记:%
	\cite{test-en},\parencite{test-zh}、\supercite{test-en, test-zh}。

	若要避免章末空白页,请在调用 pkuthss 文档类时加入 \texttt{openany} 选项。

	如果编译时不出参考文献,
	请参考 \texttt{texdoc pkuthss}“问题及其解决”一章
	“其它可能存在的问题”一节中关于 biber 的说明。
}

\begin{document}
	% 以下为正文之前的部分,默认不进行章节编号。
	\frontmatter
	% 此后到下一 \pagestyle 命令之前不排版页眉或页脚。
	\pagestyle{empty}
 
	% 自动生成标题页。
	\maketitle
	% 版权声明;因封面要求单面打印,故新开右页。
	\cleardoublepage
	% vim:ts=4:sw=4
%
% Copyright (c) 2008-2009 solvethis
% Copyright (c) 2010-2015 Casper Ti. Vector
% All rights reserved.
%
% Redistribution and use in source and binary forms, with or without
% modification, are permitted provided that the following conditions are
% met:
%
% * Redistributions of source code must retain the above copyright notice,
%   this list of conditions and the following disclaimer.
% * Redistributions in binary form must reproduce the above copyright
%   notice, this list of conditions and the following disclaimer in the
%   documentation and/or other materials provided with the distribution.
% * Neither the name of Peking University nor the names of its contributors
%   may be used to endorse or promote products derived from this software
%   without specific prior written permission.
%
% THIS SOFTWARE IS PROVIDED BY THE COPYRIGHT HOLDERS AND CONTRIBUTORS "AS
% IS" AND ANY EXPRESS OR IMPLIED WARRANTIES, INCLUDING, BUT NOT LIMITED TO,
% THE IMPLIED WARRANTIES OF MERCHANTABILITY AND FITNESS FOR A PARTICULAR
% PURPOSE ARE DISCLAIMED. IN NO EVENT SHALL THE COPYRIGHT HOLDER OR
% CONTRIBUTORS BE LIABLE FOR ANY DIRECT, INDIRECT, INCIDENTAL, SPECIAL,
% EXEMPLARY, OR CONSEQUENTIAL DAMAGES (INCLUDING, BUT NOT LIMITED TO,
% PROCUREMENT OF SUBSTITUTE GOODS OR SERVICES; LOSS OF USE, DATA, OR
% PROFITS; OR BUSINESS INTERRUPTION) HOWEVER CAUSED AND ON ANY THEORY OF
% LIABILITY, WHETHER IN CONTRACT, STRICT LIABILITY, OR TORT (INCLUDING
% NEGLIGENCE OR OTHERWISE) ARISING IN ANY WAY OUT OF THE USE OF THIS
% SOFTWARE, EVEN IF ADVISED OF THE POSSIBILITY OF SUCH DAMAGE.

\chapter*{版权声明}
\thispagestyle{empty}

任何收存和保管本论文各种版本的单位和个人,
未经本论文作者同意,不得将本论文转借他人,
亦不得随意复制、抄录、拍照或以任何方式传播。
否则一旦引起有碍作者著作权之问题,将可能承担法律责任。

% 若需排版二维码,请将二维码图片重命名为“barcode”,
% 转为合适的图片格式,并放在当前目录下,然后去掉下面 3 行的注释。
%\vfill\noindent
%\includegraphics[height = 5em]{barcode}



	% 此后到下一 \pagestyle 命令之前正常排版页眉和页脚。
	\cleardoublepage
	\pagestyle{plain}
	% 重置页码计数器,用大写罗马数字排版此部分页码。
	\setcounter{page}{0}
	\pagenumbering{Roman}

	% 中英文摘要。
	% vim:ts=4:sw=4
% Copyright (c) 2014 Casper Ti. Vector
% Public domain.

\begin{cabstract}
很多现代软件开发工具都涉及对代码的修改,例如程序修复工具、程序迁移工具等等。我们把这类工具统称为\emph{程序编辑工具}。
同时,许多程序都需要使用C预处理器,比方说C,C++和Objective-C开发的程序。由于预处理器的处理十分复杂,
在有C预处理器的情况下,许多程序编辑工具无法给出可靠的操作。
甚至有的程序编辑工具直接放弃处理未预处理的代码。

本文中我们提出了一种轻量级的双向C预处理器算法。它可以帮助程序编辑工具免去处理C预处理器的烦恼。程序编辑工具现在只需在预处理后的代码上做出修改。而我们的双向预处理器可以自动地把修改反向映射到预处理前的代码上。我们从理论上讨论了我们的双向变换满足应具备的五条性质。我们在Linux内核代码上生成修改操作并通过实验验证了我们的方法的可行性和有效性。实验结果显示我们的方法在现实项目中总能给出正确的修改并尽量不破坏源代码的完整性。
\end{cabstract}

\begin{eabstract}
Many tools directly change programs, such as bug-fixing tools, program migration tools, etc. We call them \emph{program-editing tools}. On the other hand, many programing use the C preprocessor, such as C, C++, and Objective-C. Because of the complexity of preprocessors, many program-editing tools either fail to produce sound results under the presence of preprocessor directives, or give up completely and deal only with preprocessed code.

In this paper we propose a lightweight approach that enables program-editing tools to work with the C preprocessor for (almost) free. The idea is that program-editing tools now simply target the preprocessed code, and our system, acting as a bidirectional C preprocessor, automatically propagates the changes on the preprocessed code back to the unpreprocessed code. The resulting source code is guaranteed to be correct and is kept similar to the original source as much as possible. We have evaluated our approach on Linux kernel with a set of generated changes. The evaluation results show the feasibility and effectiveness of our approach.
\end{eabstract}


	% 自动生成目录。
	\tableofcontents

	% 以下为正文部分,默认要进行章节编号。
	\mainmatter
	% 序言。
	% TODO:文章安排
	% vim:ts=4:sw=4
% Copyright (c) 2014 Casper Ti. Vector
% Public domain.

\specialchap{序言}
% 中文测试文字。

\section{Introduction}
\label{sec:intro}

% \begin{itemize}
% \item program editing tools
% \item preprocessors
%   \begin{itemize}
%   \item used by mainstream languages, including OO language C++ and
%     ObjectiveC
%   \item casual uses
%   \end{itemize}
% \item program editing tools meet preprocessors
% \item reality: program editing tools do not support preprocessors
% \item existing work: C preprocessor in refactoring
%   \begin{itemize}
%   \item too complicated
%   \item not general solution
%   \end{itemize}
% \item contributions
% \item outline of the paper
% \end{itemize}

现在,许多程序分析工具都涉及代码修改功能。
在这些工具中,有许多都是代码修复工具~\parencite{le2012genprog,le2012systematic,QiMLDW14,nguyen2013semfix}。
通常来说,修复工具的输入是一段代码和一组测试,并不断修改代码直至代码能通过测试。
另一些程序分析工具是API升级工具~\parencite{li2015swin,Padioleau06,Meng:2011}。
当API升级时出现了不兼容情况时,这些工具可以自动更新相应的API调用让程序与API契合。
我们把这类直接修改代码的工具称作\emph{程序编辑工具}。

另一方面,许多程序语言的实现都带有预处理器~\parencite{ernst2002empirical,kohlbecker1986hygienic,lee2012marco}。
最常见的预处理器是C预处理器。
许多程序语言也接受C预处理器,包括C,C++和Objective-C。
同时,程序员也时常使用C++来写一些零散的小工具。这时就会使用到预处理器。
例如Korpela~\parencite{Korpela2000}曾在文章中描述过用C++写一个HTML编辑工具:
这个工具会把页面间相同的HTML代码转换成C的宏,而不是直接生成HTML页面。
然后页面再利用这些宏最终生成HTML文件。

然而,程序编辑工具通常不会去修改程序的预处理指令。
但是这并不代表他们能够处理有预处理器的程序。
只有能够把修改反映射到预处理之前的代码的工具才算有用。
仅仅在预处理后的代码中修复错误会导致原有程序再次编译的时候错误依然存在,这样毫无意义。
这个问题具有挑战性,因为工具必须同时能理解预处理命令和目标程序语言,同时保证修改在两边能保持一致。
事实上,现有的程序编辑工具往往无法正确处理预处理指令、或是直接不处理预处理指令,例如现有的C语言工具:GenProg~\parencite{le2012genprog,le2012systematic},RSRepair~\parencite{QiMLDW14},和
SemFix~\parencite{nguyen2013semfix}。
这三个工具都只在预处理后的代码上工作。
用户需要手动检查预处理后的代码变化,并自行修改源代码——而这又增加了新bug的可能。

代码重构是一个密切相关的领域~\parencite{McCloskey:2005,Garrido2013}。
在代码重构中,程序编辑工具有时需要直接修改预处理指令。
比如:用户有时想重命名一个宏,或者需要提取一个宏作为重构的一部分。
在这种情况下,工具开发者别无选择,只好修改预处理指令。
典型情况中,工具开发者会定义一种新的C语法使得原有的C语法和预处理指令能兼容。
但是,如果我们考虑更一般的程序编辑工具,这种方法就捉襟见肘了。
首先,工具开发者需要在真正设计工具之前把精力花费在学习语言的细节上。
其次,学习新语言的努力并不能在其他语言中复用。

本文中我们提出了一个轻量级的支持C预处理器的程序编辑工具实现方法。
该系统是一个双向C预处理器:
原有的预处理过程可以背看作一个正向变换,考虑程序编辑工具对预处理后的代码做出修改
双向C预处理器可以把这些修改反向映射回源代码。
于是,程序编辑工具现在可以只关注于预处理后的代码,而不需要考虑预处理器带来的影响,
并把映射修改的工作交给我们的自动工具\footnote{这个过程并不是全自动的。
因为我们的工具暂时只支持程序编辑的基本操作。
尽管任意程序修改步骤都可以用通用代码差分方法转换成基本的修改操作~\parencite{fluri2007change},
但是如果工具能直接提供基本编辑操作可以有最好的效果}。



%This idea is
%based on the observation that, unlike refactoring tools that often
%have to directly manipulate the preprocessor directives, many
%program-editing tools are not directly related to preprocessor
%directives and can be implemented more conveniently with only
%preprocessed programs.

在这里我们列举一些例子:
(1)上文中所提到的三个学术届认可的错误修复系统现在可以处理预处理前的代码;
(2)API升级软件现在可以在有预处理代码的情况下更好地实现;
(3)所有并不需要关心预处理过程的程序编辑工具都能够被改善。


如果要实现一个双向预处理器,首先会想到双向变换技术。
虽然现在存在着几种双向变换的技术~\parencite{MaHNHT07,Voigtlander09bff,voigtlander2010combining},但是它们都是为数据的转换设计的。
给定一个源数据集$s$,一个变换程序$p$,和一个目标数据集$t=p(s)$,这些方法试图将$t$上的变化描述成$s$的变化。
然而,C的预处理器与数据的转换不同,因为C的源代码不仅包含了作为数据的代码,还包含了作为变换程序的预处理指令。
这就要求反向变换的机制能处理更复杂的情况:当目标数据发生变化时,我们可能要变化源数据、转换程序、或者是二者都变化。

%It is not easy to bidirectionalize CPP. 

本系统的一个创新处是在能够处理上文提到的复杂双向变换情况的同时,尽量最小化对变换程序$p$的修改。
首先,该设计中从不引入新的宏定义或修改现有的宏定义,有效地限制反向变化的影响。
其次,该设计只会移除/修改宏调用而并不会创造新的宏调用。
再有,该设计只会在必要的时候移除宏调用。这样,我们就能尽可能保证源代码和修改之前的相似度。


实现这样的算法设计也是十分具有挑战的。
典型的这类双向变换的方法~\parencite{MaHNHT07,Voigtlander09bff,MMHT10}是顺着程序的抽象语法树AST拆分双向变换。
每一个子树对应着一个小的双向变换,组合起来就成为整个程序的双向变换。
但是,一个未预处理的C++的程序并不能简单地解析成树状结构。
比方说,在下列代码中,
\begin{lstlisting}
#define inc(x) 1+x
#define double(x) 2*x
inc(double) 2
inc(double) (x)
\end{lstlisting}
第一个宏调用\code{inc(double)}会自动展开成为一个单独的语句($statement$)。
但是第二个宏调用不能单独展开成一个语句。它需要连上之后的\code{(x)},循环展开后才算完整的句子。
因此,我们不能把第二个宏调用\code{inc(double)}当作独立的一段并直接对他做双向变换的分析。
为了克服这个困难,我们为描述类似C预处理程序的情况设计了新的模型。
我们把预处理看作是对代码数据的重写规则(\emph{rewriting rules})集,而不是直接把代码解析成抽象语法树。
这样的模型把程序的双向变换看作是重写规则的双向变换。


另外,该设计也可以被证明满足双向变换的正确性:
(1)如果预处理后的程序没有变化,那么源代码也不会变化;
(2)源代码在接受了反向映射的变化后,那么新的源代码预处理后和接受了修改操作的预处理后代码是等价的。
这两个性质在双向变化领域被称作GETPUT和PUTGET~\parencite{Foster:2007}。

总结一下,该项目的贡献如下:
\begin{itemize}
% \item We propose a lightweight approach to support handling of the C
%   preprocessor in program-editing tools. Our approach bidirectionalizes
%   CPP, reflecting the changes on the preprocessed code
%   back to the unpreprocessed code.
% \item We demonstrate that our approach satisfies the two basic properties
%   of bidirectional transformation, GETPUT and PUTGET, in the context
%   of CPP.
\item 我们提出了一个轻量级的在程序编辑工具方面的双向C预处理器。
	我们分析了不同可能的设计并提出了五个反向变换应有的性质,
	包括GETPUT和PUTGET
	(\secref{sec:problem})。
\item 我们提出了一个能够符合五个性质的算法。
	该算法把C的预处理看作是重写规则的集合,
	并结构性地把预处理的双向变换转换到重写规则的双向变换上
	(\secref{sec:approach})。
\item 我们在Linux内核上验证了我们的方法,并和另外两种基本的
	做法进行比较:一个是直接把整个修改的文件映射回去;另一个是只把修改了的代码行映射回源代码。
	实验的结果显示相较于其他方法,我们的方法破坏了相当少的宏调用,
	并且总是可以给出正确的修改,而其他方法有时不行(\secref{sec:evaluation})。
\end{itemize}

本文接下的部分组织如下:
首先我们会在~\secref{sec:problem}中描述项目背景和需要解决的问题。
接着我们会在~\secref{sec:approach}中提出我们的算法,证明它满足双向预编译器的五种性质。
然后我们会在~\secref{sec:evaluation}中给出我们的实现与实验的过程、细节与结果,并进行讨论分析。
最后我们会在~\secref{sec:related}中讨论相关工作,并在~\secref{sec:conclusion}中总结。


% The rest of the paper is organized as follows. \todo{finish this}.

%Finally, we discuss related work in %\secref{sec:related} and conclude
%the paper in %\secref{sec:conclusion}.

	\chapter{背景介绍}
\label{sec:problem}



% \begin{itemize}
% \item what are C macros
% \item an example about why C macros are difficult
%   \begin{itemize}
%   \item needs to the choice for insertion
%   \item needs to show concatenation and single \#
%   \item needs to show high-order macros
%   \item needs to show \#if and other directives
%   \end{itemize}
% \item explain the complexity of C macros
% \item explain the complexity of propagating back
%   \begin{itemize}
%   \item the choice for insertion
%   \item dealing with concatenation
%   \end{itemize}
% \end{itemize}

% \begin{itemize}
% \item A table of C preprocessor directives
% \item An example to show the process of C preprocessing
% \item A change example (introduce a guarded condition) to discuss the bidirectional behavior
%   \begin{itemize}
%   \item show that changing macro definition is inappropriate
%   \item show that introducing new macro invocation is inappropriate
%   \end{itemize}
% \item More change examples
%   \begin{itemize}
%   \item change the variable name
%   \item break only the outer level
%   \item copying
%   \end{itemize}
% \item trivial solution 1: per file
%   \begin{itemize}
%   \item break a lot of macros
%   \item removing macro definitions
%   \end{itemize}
% \item trivial solution 2: per line
%   \begin{itemize}
%   \item still breaks useful macros( in the second example, or in
%     copying)
%   \item may produce erroneous result when a macro invocation spans
%     multiple lines
%   \end{itemize}
% \end{itemize}

\section{C预处理器} \label{sec:CPreprocessor}

\begin{table*}[htbp]
  \small
  %\centering
  \caption{主流预处理器指令与操作}
  \label{tab:preprocessor}
  \begin{tabular}{|p{3cm}|p{3cm}|l|p{4cm}|}
    \hline
    Directives & Functionality & Example & Result\\
    \hline
    \hline
    \code{\#progma} & Compiler options & \begin{lstlisting}
#progma once
\end{lstlisting} & removed from the preprocessed file\\
    \hline
    \code{\#include} & File Inclusion & \begin{lstlisting}
#include <stdio.h>
\end{lstlisting} & the content of "stdio.h"\\
    \hline
    \code{\#if, \#ifdef, \ldots} & Conditional compilation & \begin{lstlisting}
#ifdef FEATURE1
  x = x + 1;
#endif
\end{lstlisting} & \code{x = x + 1;}\\
    \hline
    \code{\#define X} & Object-like macro definition & \begin{lstlisting}
#define X 100
a = X;
\end{lstlisting} & \code{a = 100;}\\
    \hline
    \code{\#define X(a, b)} & Function-like macro definition & \begin{lstlisting}
#define F(x) x * 100
F(10);
\end{lstlisting} & \code{10 * 100;}\\
    \hline
    \code{a \#\# b} & Concatenation & \begin{lstlisting}
#define X a_##100
X
\end{lstlisting} & \code{a\_100}\\
    \hline
    \code{\#b} & Stringification & \begin{lstlisting}
#define F(x) #x;
F(hello);
\end{lstlisting} & \code{"hello";}\\
\hline
\code{\_\_FILE\_\_}, \code{\_\_DATE\_\_}, \ldots & Predefined macros &
                                                                       \code{\_\_FILE\_\_}
                                         & main.c \\
                                        
\hline
    
                                         
  \end{tabular}
\end{table*}

表~\ref{tab:preprocessor}显示了主流预处理器的指令与操作。
一条预处理指令在行首以\#开头,在行末结束。
宏可以被预处理指令定义,但是它们自身并不是预处理指令。
本质上,我们认为现在有四种主要的预处理指令:
\code{\#progma}~提供了编译选项,\code{\#include}~描述了包含的头文件,
\code{\#if}~提供了条件编译选项,\code{\#define}~是宏定义指令。
另外,在一个宏定义中,我们可以使用类似\code{\#\#}~和\code{\#}~的指令来连接两个变量、或字符化一个变量。
最后,还存在一些预定义的宏,如~\code{\_\_FILE\_\_}~,会随着上下文的不同而被替换。

当C预处理器处理一个源文件的时候,它会依据以下的方法来转换源程序文件:
\begin{itemize}
\item 首先展开~\code{\#include}~和~\code{\#if}~指令,然后再重复扫描展开后的代码词序列
\item 对于每个宏调用,预处理器先处理参数的展开,然后再展开宏调用。
\item 对于含有 \# 和 \#\#的参数,预处理器并不会处理这类参数。相反,预处理器会把参数直接文本拷贝到展开项中。
\item 当一个宏调用被展开之后,这个被展开的程序词序列会被再次扫描。
      如果这时还有宏没有展开,预处理器会把宏调用展开。
\item 为了避免循环展开宏调用,如果一个宏定义已经在展开过程中被展开,那么它就不会被再次展开。
\item 如果在展开宏的过程中生成了新的预处理指令,该指令并不会被预处理器执行。
\end{itemize}

\begin{figure}[ht]
  \centering
\begin{lstlisting}
#define SAFE_FREE(x) if (x) vfree(x);
#define FREE(x) vfree(x);
#define RESIZE(array, new_size, postprocess) \
  g_resize_times++; \
  postprocess(array); \
  array = vmalloc(sizeof(int)*(new_size));
#define GARRAY(x) g_array##x;

RESIZE(GARRAY(2), 100, FREE);
\end{lstlisting}
  \caption{一个预处理的例子\label{fig:example} }
\end{figure}

这里举一个实例,让我们来考虑在图~\ref{fig:example}里的代码片段。
这个例子向我们展示了许多实际项目中的宏定义与宏调用。
这段代码中含有四段宏定义和一个宏调用。
前两个宏定义被包含在一个用户自定义的空间释放函数里。
第三个宏定义是为了用户自定义的内存空间管理和日志记录而重新调整数组的大小。
最后一个宏是为了定义一组特殊的全局变量。
当预处理器扫描这段代码时,第一个参数~$RESIZE$~将会被处理。
此时,~\code{GARRAY(2)}~会被展开成~\code{g\_array2}~。
尽管第三个参数~\code{FREE}~已经被定义成一个函数状的宏(\emph{function-like macro}),
但是并没有能够提供给~\code{FREE}~的参数,因此此时预处理器并不会把它当作一个宏调用来处理。
然后~\code{RESIZE}~的宏调用被展开,于是我们得到了以下的代码:
\begin{lstlisting}
g_resize_times++;
FREE(g_array2);
g_array2 = malloc(sizeof(int)*(100));
\end{lstlisting}
现在我们能看到在展开的宏中,我们已经给~\code{FREE}~提供了一个参数列表。
因此接着系统会展开~\code{FREE(g\_array2)}~,我们得到以下代码:
\begin{lstlisting}
g_resize_times++;
vfree(g_array2);
g_array2 = malloc(sizeof(int)*(100));
\end{lstlisting}
换句话说,\code{RESIZE}~实际上是一个高阶宏(\emph{high-order macro}),
因为他的第三个参数也是宏。

\section{反向变换的设计}\label{sec:backdesign}
现在让我们来考虑一下输入为预处理后代码的程序编辑工具们。
比方在下面的例子中,程序编辑工具会发现~\code{vfree}~可能存在内存泄漏的可能,
所以工具会在这句代码前加入一个保护语句,如下:
\begin{lstlisting}
g_resize_times++;
if (g_array2) vfree(g_array2);
g_array2 = malloc(sizeof(int)*(100));
\end{lstlisting}

接下来的任务是生成反向变换。反向变换一般来说输入是预处理后代码上的修改,
然后产生预处理前代码上的修改。
当生成的修改作用在预处理前的代码上后,新的代码在预处理后会得到与作用输入修改的预处理后代码相同的结构。
对于之前例子里面程序编辑工具做出的修改,我们有两种处理办法:
(1)我们可以修改~\code{RESIZE}~的宏定义,把加入的~\code{IF}~语句加入到宏定义中;
(2)我们可以展开~\code{RESIZE}~的宏调用,并按照预处理后的修改把保护语句添加到原程序中。
第二个选项只影响到了这一小段局部的代码,而第一个选项有可能会影响到全局的其他宏调用。
但是,因为反向变换并不知道应该把修改作用到局部还是全局,一个更安全的做法是选择只影响局部的代码。
在本例中,有很大可能我们并不需要对每一个$vfree$的调用都加上保护语句,因为这样会带来大量不必要运行时间。
我们有理由相信程序编辑程序会根据需要选择是否在$vfree$前加上保护语句。
进一步讲,局部的选项会尽可能小地修改原有代码,因为全局会影响程序的许多部分。
这些可能性让我们想到一个双向预处理器的第一个性质。

\begin{decision}
反向变换不应该改变任何宏定义。
\end{decision}

根据我们定义的第一个性质,我们应该展开宏调用后再在源代码中加入保护语句。
然而现在在映射修改时我们又面临着以下两个展开的选择。我们可以把所有的宏都依次展开:

\begin{equation}
\begin{minipage}{7.6cm}
\begin{lstlisting}
g_resize_times++;
if (g_array2) vfree(g_array2);
g_array2 = malloc(sizeof(int)*(100));
\end{lstlisting}
\end{minipage}
\label{eqn:expandall}
\end{equation}
或者我们也可以只把宏展开一层:
\begin{equation}
\begin{minipage}{7.6cm}
\begin{lstlisting}
g_resize_times++;
if (GARRAY(2)) FREE(GARRAY(2));
GARRAY(2) = malloc(sizeof(int)*(100));
\end{lstlisting}
\end{minipage}
\label{eqn:goal}
\end{equation}
\newcommand{\coderef}[1]{code piece~(\ref{#1})}
我们认为代码段~\ref{eqn:goal}比代码段~\ref{eqn:expandall}更好,因为它保存了更多原有的结构,
这使得代码更加易懂,重用或是维护。
这就引入了我们的第二条性质。

\begin{decision}
反向变换应该尽可能保存现有的宏调用。
\end{decision}

也许有人会提议我们进一步地把这个保护语句缩减成一个新的宏,
或者复用现有的某一个宏来实现保护语句的功能。
在本例子中,我们的保护语句被~\code{SAFE\_MACRO}~这个新宏定义包含。
代码可能如下:
\begin{lstlisting}
g_resize_times++;
SAFE_FREE(GARRAY(2));
GARRAY(2) = malloc(sizeof(int)*(100));
\end{lstlisting}
但是,这样做是十分危险的。因为宏定义并不是用来替换所有语义相同的代码片段。
比方说,Ernst等人~\parencite{ernst2002empirical}就在文章中描述过某一个宏定义,
\lstinline!#define ISFUNC 0!, 定义了一个在系统调用中时常用到的常数。
很明显,我们并不能把整个系统里的~\code{0}都替换成~\code{ISFUNC}。
这就引入了我们的第三个性质。

\begin{decision}
  反向变换不应该引入新的宏调用。
\end{decision}

除了已经提到的三个性质之外,我们还有两条从双向变换领域借鉴过来的性质,
叫做 GETPUT 和 PUTGET~\parencite{Foster:2007}。
令$s$是预处理前的源程序,$t$是预处理后代码,$c_t$是作用在$t$上的变化,
$c_s$是完全的反向变换所提供的作用在$s$上的变化。

\begin{decision}[GETPUT]
  如果$c_t$是空的,那么$c_s$也是空的。
\end{decision}

\begin{decision}[PUTGET]
  令$s'=c_s(s)$为新的预处理前的源代码。意为把生成的$c_s$作用在$s$上。
  令$t'=c_t(t)$为作用了修改操作的预处理后代码。
  对$s'$做预处理会得到$t'$。
\end{decision}

% \begin{decision}
%   The backward transformation fails to produce $c_t$ if and only if
%   there exists no $c_t$ satisfying the above five requirements.
% \end{decision}

以上五条性质一起定义了反向变换的行为:它应该通过
修改宏调用参数、修改普通代码、展开宏调用并且展开尽可能少的宏调用
的方法来把预处理后的修改映射到源代码上。其中性质4与性质5又被认为是双向变换的正确性定义。~\parencite{Foster:2007}

\section{简单的方法}\label{sec:naive}
我们将在本节中讨论为什么简单的方法并不能满足我们提出的五条性质。\\

\noindent\emph{简单的方法 I (per-file).}
第一种简单的方法是直接把修改过的文件不加处理地拷贝覆盖源代码。
这种方法十分容易实现,我们称之为\emph{per-file},但是他有两大缺点。
首先,原有的未预处理源代码可能含有宏定义,然而现在的做法会让这些宏定义都丢失掉,
那么在其他地方,比如包含了该文件并调用了相应宏的其他代码,可能就会出现错误。
其次,这会导致把文件中所有的宏调用都展开,甚至包括所有的~\code{\#include}~指令。
整个代码面目全非,破坏了完整性。

\noindent\emph{简单的办法 II (per-line).}
第二个简单的办法是利用代码的性质,只把预处理后代码中被修改的行反向映射回源代码。
我们把这个方法称作\emph{per-line}。这个做法看起来是可行的,因为现代的预处理器
会记录下预处理前后行间的对应关系。
比方说,当GCC预处理我们之前的例子时,它会把1-7行替换成空行。
同时它会把从第9行开始的几行展开压缩成一行。
可以看到,现代预处理器都记录下了源代码与预处理代码行间的一一映射。

虽然第二种方法相较于第一种有了不删除宏定义的好处,它依然存在隐患。
首先,依然有不少宏被不必要地展开了。
在我们的例子中,如果我们把修改的那一行复制回去,那么代码段~\ref{eqn:expandall}就会时结果,
但这与我们定义的第二条性质不符。
甚至,如果我们考虑有的工具会在代码中挑选代码复制插入来修改,比如GenProg~\parencite{le2012genprog}
就会在代码中拷贝不同地方的代码来修改程序的错误。
这样一来拷贝的行中所有的宏调用都会被展开,代码的完整性还是被破环。
另外,这还并不是最严重的。
如果源文件中的宏调用是一个多行宏(在我们的调研与实验中也确实发现了这样的情况\secref{sec:evaluation}),
那么只替换宏调用的第一行会直接带来错误的结果,反而引入了新的bug。
比方说,如果在源代码的宏调用中插入了一个断行,如下:
\begin{lstlisting}
RESIZE(GARRAY(2),
  100, FREE);
\end{lstlisting}
在GCC中,这个宏调用会被展开成两行。其中第一行时全部的宏展开语句,而第二行是空行。
因此,修改操作只会作用于第一行。
如果反向变换仅仅把第一行替换成新的代码,那么就会留下不正确的程序。



%%% Local Variables: 
%%% mode: latex
%%% TeX-master: "main"
%%% End: 

	\chapter{算法设计}
\label{sec:approach}
正如之前提到的,我们算法的基本思想是把C预处理器当作一组重写规则,
而反向变换就是这些规则相应的逆向规则。
在本章中,我们会先描述本项目的C预处理器的模型(\secref{sec:forward})。
然后我们会描述该系统所支持的修改操作。
接着我们会集中阐述其中第一种操作:替换操作的处理方法(\secref{sec:changes})。
因为我们需要吧每一条重写规则反向应用,因此我们需要记录下预处理时使用
重写规则的顺序(\secref{sec:changes}),然后按照顺序依次为这些规则生成反向变换(\secref{sec:steps})。
我们也会讨论这些步骤/过程为何能满足我们之前讨论的双向预处理器的五条性质(\secref{sec:correctness}),
并且给出一个更优化的算法。
最终,我们会讨论如何把不同类型的修改操作都转换成替换操作(\secref{sec:extend-other-changes})。


% We introduce our
% approach in the following steps. First, we present our model of
% forward preprocessing, which forms the basis of the correctness
% discussion (\secref{sec:forward}). Second, we introduce a model for
% encoding changes on the preprocessed code (\secref{sec:changes}) and
% describe how to backward transform one type of changes: replacement of
% a token
% (\secref{sec:backward}). % We introduce two core concepts, rewriting step and
% % rewriting action, for recording the process of a forward
% % transformation and perform the backward transformation.
% The backward transformation is based on tracing the forward
% transformation as rewriting steps (\secref{sec:steps}).
% We also
% discuss how this process satisfies the requirements and laws (\secref{sec:correctness}). Finally, we
% discuss how to convert other types of changes into replacement (\secref{sec:extend-other-changes}).

为了读者可以更快理解我们算法设计的思想,我们暂时只考虑C预处理器指令的一个子集:
去除 \# 操作和 \#\#操作,去除~\code{\#include}~操作,也不考虑宏出现循环调用的情况。
我们会在之后的章节中(\secref{sec:fullC})讨论如何把子集上的模型扩充到
支持全部C预处理指令的完整模型。

\newcommand{\dstart}{\ensuremath{\langle\#}\xspace}
\newcommand{\dend}{\ensuremath{\rangle}\xspace}
% \newcommand{\env}{\ensuremath{CTX}\xspace}

\section{模拟正向变换:预处理}\label{sec:forward}
我们把C预处理程序需要处理的程序看作词(\emph{token})的一个序列。
为了从词序列中识别出预处理指令,我们依赖于两个特征词:
行首的\#符号和该行最后的换行符。
我们同时也假设当前环境中所有定义的宏都存储在环境变量\emph{$context$}里。

我们把C预处理器语法看作是重写规则的一个集合。
每个重写规则都有$guard \hookrightarrow action$这样的形式。
当$guard$是真时,$action$会把当前词序列的前几项替换成指定的词,
然后在替换的位置后继续下一条替换指令。

在模型中,$guard$ 和 $action$ 都可以被看作是函数。单独来看,
$guard$ 函数的输入为当前剩下的词序列和当前的上下文环境 $context$,
它将会输出一个表示是否要把当前规则应用到现在的词序列的布尔值。
与此同时,$action$ 函数把当前还剩下的词序列,上下文环境当作输入,然后生成
一个四元组 $(finalized, changed, 
restIndex, newContext)$。
这个四元组中的变量含义如下:
$restIndex$ 表示在这变量之前的词都已经被重写规则处理过了。
在这些被处理过的词序列中,$finalized$ 表示了一串不需要再应用规则的词序列,
而 $changed$ 是可能还需要被扫描的序列。
而最后一个变量 $newContext$ 代表着更新过的上下文环境。

有了这些定义后,我们算法中的正向变换部分在算法~\ref{alg:forward}中展示。
该算法循环应用规则 $R$ 直至整个词序列都被处理。


% , where $guard$ is a condition to
% determine whether this rule can be applied to the head of the current token
% sequence, and $action$ takes a subsequence from the beginning of the
% sequence, replaces it with another sequence, and separates the
% replaced sequence into 
% into a finalized sequence and the remaining sequence, where the
% finalized sequence do not need to be further scanned and the remaining
% sequence needs to be scanned. A forward transformation iterates
% all rules, and applies the first applicable rule until the no more
% token need to be scanned.

\begin{algorithm}
  \newcommand\mycommfont[1]{\rmfamily{#1}}
  \SetCommentSty{mycommfont}
  \caption{正向预处理算法  \label{alg:forward}}
  \KwIn{token sequence $src$, rule list $R$}
  \KwOut{new token sequence $res$}
  $ctx \leftarrow \{\}$\;
  \While{$src.length > 0$}{
    \For{$r \in R$}{
      \If{$r.guard(src, ctx)$} {
        $(finalized, hanged, rest, ctx') \leftarrow r.action(src, env)$\;
        break\;
      }
    }
    $res \leftarrow res + finalized$\;
    $src \leftarrow changed + src.sub(rest)$\;
    \tcp{$sub(l)$ returns a
      subsequence starting from $l$} 
    $ctx \leftarrow ctx'$\;
  }
 % {\small Function $guard$ takes the remaining token sequence and the current definitions as
 % input, and returns a Boolean value to denote whether the rule can be
 % applied. $action$ takes the token sequence and the definitions
 % as input, and returns a token sequence that needs not be scanned, a
 % token sequence needs to be further scanned, and an updated set of definitions.}
\end{algorithm}

在C预处理情况中,规则列表$R$中总共有四个规则。
其中一个处理条件预处理指令,例如$\#if$, $\#ifdef$;
一个处理其他的预处理指令,这样我们可以用它来清除预处理指令并且更新上下文环境;
一个处理宏调用;
最后一个处理普通字符文本。
这四个规则如下定义:

\begin{itemize}
\item 规则1: 这个规则处理条件编译选项。$guard$函数将判定当前词序列的开头是否是一个独立
  的预处理条件指令,例如$\#if$,$\#ifdef$。如果为真,
  $action$函数首先会在当前上下文环境中检验条件选项是否为真,然后根据是否为真选择使用
  真值分支或假值分支编译。
  选择分支后,用该分支替换原有的指令,并把替换成功的新词序列记录在$changed$里。
  而$finalized$是空的。
\item 规则2: 这个规则处理其余所有预处理指令。$guard$函数将判定当前词序列的开头是否是一个
  独立的预处理指令。如果为真,
  $action$函数会解析这条指令,并对当前上下文环境做出必要的调整,比如添加一条宏定义。
  最后,该指令之后的词下标将会被记录成$restIndex$,而$finalized$和$changed$为空。
\item 规则3: 这条规则会展开宏调用。$guard$函数将判定当前词序列的开头第一个词是否是一个
  对象宏调用(\emph{object-like macro})或者当前词序列的开头前两个非空白词是否是是一个
  函数宏调用(\emph{function-like macro})和一个开括号。如果为真,
  $action$函数会做以下两个步骤的操作:
  \begin{itemize}
  \item 首先,我们会用规则3和规则4循环调用处理参数\footnote{在C预处理器中未定义
    把预编译指令放在参数中的操作~\parencite{CStandard}。在次我们认为在
    参数中不存在预处理指令}。
  \item 然后,我们把宏展开中各个参数出现的位置都替换成已经处理完了的宏参数。
    整个展开的部分都被记录在$changed$里,而$finalized$是空的,$restIndex$记录
    了宏调用之后的第一个词的位置。
  \end{itemize}
\item 规则4: 这个规则处理那些没有被其余规则处理的普通文本词。$guard$函数总是返回真。
  $action$函数会把词序列中的第一个词放入$finalized$中,返回一个空的$changed$,
  并把下一个词的下标标记为$restIndex$。
\end{itemize}

让我们来看一个例子来理解这些规则的应用。考虑以下程序。
\begin{equation}
\begin{minipage}{0.4\columnwidth}
\begin{lstlisting}
#define x 100
hello x
\end{lstlisting}
\end{minipage}
\label{eqn:smallexample}
\end{equation}
% To preprocess this program, we recursively apply the rules over the
% program. 
当我们的系统处理这段代码时,第一个能应用的规则是规则2。
它会把第一个宏定义解析出来,存储在上下文中,并把接下来的下标移动到下一行。
这时剩余的词序列为~\code{hello x}。
接着应用规则4,并把~\code{hello}移动到$finalized$的序列中。
然后应用规则3,把宏调用 \code{x} 展开成 \code{100}。
最终,系统将会再次扫描一遍 \code{100} 并使用规则4把它移动到$finalized$序列中。
当剩余词序列为空时,整个处理过程就停下了。

% From the definitions of the rules we can prove a simple property. For
% any rule $r$, let $r.guard(src, c)$ be true and $r.action(src, c) = (f, h, restIndex, c')$. Supposing
% $src'$ differs from $src$ only for the tokens at or after $restIndex$,
% we know that $r.guard(src, c)=true$ and $(f, h, restIndex, c') = r.guard(src', c)$. In other
% words, if we change the 

\section{模拟修改}\label{sec:changes}
程序编辑工具可以通过各种各样的方式对一段程序进行修改。
本文中我们考虑三种基本的修改:替换、插入和复制。
这三种基本的修改是我们在分析了现有的主流代码修复工具
例如GenProg~\parencite{le2012genprog} 和 SemFix~\parencite{nguyen2013semfix}
之后总结而成。
这些代码修复工具往往会拷贝或创造一个语句来替换现有语句或插入来修改程序错误。

这三种操作都直接地对词序列进行操作。
一个替换操作描述为一个二元对$(l, s)$,其中$l$是要被替换的词的位置,
而$s$是将要替换在$l$位置的一串词序列。
一个插入操作也是一个二元对$(l, s)$,其中词序列$s$将会被插入到位置$l$
之后。
一个拷贝操作时一个三元组$(l, l_b, l_e)$。它表示从位置$l_b$(包含)起至
$l_e$(不包含)的词序列将会被拷贝插入到位置$l$的词之后。

我们可以看出替换操作涵盖的删除操作。一个形式为$(l, [])$
的替换操作就代表删除了一个词,其中$[]$表示空序列。

本章接下来的部分,我们将讨论如何实现一个可以处理替换操作的反向变换。
我们也会讨论如何把其他两种操作转换成替换操作。
% If a change set contains only
% replacements, we could also interpret the change set as a total function
% mapping each index into a token sequence, where unchanged indexes
% are mapped to the original tokens at the indexes.

\section{重写步骤}\label{sec:steps}
正如在序言中介绍过的,我们的反向变换操作会逆向实施之前提到的重写规则。
为了生成好的反向变换,我们设计了一个数据结构来记录下正向展开时
使用了哪些规则,分别应用在代码的哪个部分。
我们把这种记录的数据结构称作 \emph{重写步骤}。
每一个重写步骤都是一个四元组, $(src, i, ctx, r)$。
其中$r$表示规则$r$已经被应用到一段子序列$src$。
自序列位置在$i$,应用规则时的上下文环境为$ctx$。
其中$src$总是代表了暂时的词序列,区别于源序列与最终序列,是应用了
之前所有规则后生成的临时序列。

相应的,一个完整的正向序列会被记录成一个重写步骤序列。
比方说,正向处理之前例子中的代码段 \ref{eqn:smallexample} 时,
我们的算法会产生以下的重写步骤序列:
$(P, 0, \{\}, R2)$,$(hello\ x, 0, \{x\}, R4)$,$(hello\ x, 1, \{x\}, R3)$,和
$(hello\ 100, 1, \{x\}, R4)$。
其中$p$是源程序。

% We use a data structure to record how a rule application changes the
% token sequence, which is the key to our bidirectionalization. We call
% this data structure \emph{rewriting step}. Intuitively, we may first
% consider 
% A rewriting step is a
% triple, $(ctx, args, body)$. The first component, $ctx$, is the
% current context. The rest two steps contains two types of rewriting
% actions, argument expansions and body expansions, which are inner
% changes performed to form a rewriting step. Component $args$ contains
% a set of argument expansions, and $body$ is a body expansion.

% An argument expansion records how an argument to a macro invocation is
% expanded, and is recursively defined as a sequence of
% rewriting steps.

% A body expansion 


\section{考虑替换操作的反向变换}\label{sec:backward}
反向操作的基本思想是把预编译后代码上的修改通过重写规则序列
一步步映射到预处理前的代码上。这样当我们再跑一次正向变换时,
只会有两种情况:1. 程序会按照相同的重写规则再次正向展开
2. 程序会按照等价的重写规则序列展开,这个规则序列长度可以是0。
我们设计的反向变换操作在算法~\ref{alg:backward}中。
其中$backward$函数沿着每一个重写步骤映射程序上的修改,
或者是当它发现没有合适的修改可以映射时报错。

\begin{algorithm}
  \newcommand\mycommfont[1]{\rmfamily{#1}}
  \SetCommentSty{mycommfont}
  \caption{生成反向变换算法 \label{alg:backward}}
  \KwIn{a sequence of rewriting step $rs$}
  \KwIn{a set of replacements $c$}
  Reverse sequence $r$\;
  \For{$r \in rs$}{
    $c \leftarrow backward(rs, c)$\;
    \If{$backward$ failed}{print ``Changes cannot be applied.''\; \Return{}\;}
  }
  \Return{$c$}\;
\end{algorithm}

我们从伪代码中可以看出,实现反向变换的关键在于如何实现 $backward$ 函数。
$backward$ 函数的行为随着规则的不同而改变。
在此,我们根据复杂度,从简单到复杂讨论该函数的功能。

首先,如果当前重写步骤是依据规则2处理的,这意味着此处处理了一个非条件的
预编译指令,正向预编译时,这条指令被预处理器删除。反向变换会根据预处理
指令的位置来考虑是否要偏移操作。比方说,考虑以下的代码程序:
\begin{lstlisting}
#undef hello
hello
\end{lstlisting}
现在如果我们有一个$(0, x)$的替换操作,那么在这个例子中,
$hello$会被替换成$x$。
$backward$函数会把该替换操作位移到$(4, x)$来保证当前
替换操作依然是作用在$hello$这个词上。

第二,如果当前重写步骤是依据规则1处理的,这意味着此处预处理器
处理一个条件预处理指令。预处理器当时把这一段替换成了条件指令的一个分支。
在反向变换中,我们把在分支上发生的修改操作映射到原来的条件指令中,
同时计算必要的偏移量等。
因为我们不会改变宏的定义,此处的条件预处理指令在再次预处理时,
还是会选择相同的分支进行替换。

第三,如果当前重写步骤是依据规则4处理的,这意味着此处预处理器
处理了一段非指令非宏调用的普通文本字符。正向预处理器应该把一个
词放到$finalized$里去了。
反向变换需要保证在修改过后的词序列上,依然会应用规则4。
举例来说,如果用户把$hello\ c$ 中的 $hello$ 替换成了 $a\ b$,
我们需要检查,从$a$开始,是否唯一的预处理办法就是应用两次规则4来替换当前序列。
在这个例子中,只有当规则4的$guard$函数能够在要么 $a\ b\ c$ 或者 $b\ c$上返回真才行。
这时候就会发生反向变换失败的情况。
比方说,如果程序编辑程序给出的修改时把$hello$修改成$x$,
其中$hello$并不是宏调用,但$x$是宏调用。那么此时应该报错。
另一个情况是程序编辑程序把$GARRAY\ hello$替换成了$GARRAY\ (hello)$。
此时$GARRAY$是一个含有一个参数函数式宏调用。
在这种情况下,之前$GARRAY$被当作文本使用规则4重写因为它之后并不是括号。
而在新的序列中,$GARRAY$变成了一个宏展开。此时很有可能造成程序错误。
以上两种情况$backward$函数都应该报错,自动程序并不能轻易地把修改映射到代码中去。

% ???????????

最后,如果当前重写步骤是依据规则3处理的,这意味着此处预处理器
处理了一个宏调用。这种情况是最复杂的。
规则3包含了两个步骤。我们首先试图沿着两个步骤逆向重写代码。
如果两个步骤中有一个失败了,那我们只能试着展开宏调用。

在第二个小步骤中,宏展开中各个参数出现的位置都已经用展开的参数替换了。
如果有必要,反向变换需要把作用在展开后的参数上的修改都映射到参数上去。
这时有两种情况会导致逆向重写失败,也就是说我们不得不展开宏调用:
(1)被修改的词并不是宏的参数出现的位置中的;
(2)同一个参数在多个地方出现,被多次修改,但不同地方修改的内容不一样。
比方说,考虑以下代码段:
\begin{equation}\label{eqn:expansion}
  \begin{minipage}{0.8\columnwidth}
\begin{lstlisting}
#define x 1
#define plus(a) a+a
plus(x);
\end{lstlisting}
  \end{minipage}
\end{equation}
如果我们把 $1+1$ 变换成 $1-1$ 或者 $1+2$,我们在反向时就得展开宏调用。

在第一个小步骤中,我们预处理了宏的参数。
反向变换应该循环地调用$backward$来先把参数上的修改通过反向的重写步骤应用到每个参数上。
最终,我们需要对整个代码段在映射修改时做一次安全检查:
如果被修改的参数在最后含有了一个逗号,并且这个逗号没有被单独的括号包裹住;
或者是存在括号不匹配的情况,我们还是必须展开宏调用。
这是因为一个顶层的逗号或者不匹配的括号会破环原有的参数列表结构,影响程序的正确性。

如果在上述两个反向步骤中,反向变换失败,我们不得不展开宏调用。
展开宏调用时,会产生作用在宏展开式上的相应修改操作。
举个例子,我们不妨假设在上面的代码例子~\ref{eqn:expansion}中$1+1$中的第二个$1$被替换成$2$。
此时,映射来的修改会把 $plus(x)$ 中的 $plus$ 替换成 $x+2$ 然后删除掉 $(x)$。


% To
% perform the macro expansion, we need to generate the expanded form to
% replace the macro invocation.

在展开宏调用的过程中,关键是如何构造替换序列,在刚才的例子里时$x+2$。
从正向变换我们可以看出,没有变过的预处理词序列为 $1+1$, 
而且这两个 $1$ 都来自参数 $x$。
所以我们把参数 $x$ 的重写步骤拷贝到两个 $1$ 的词上。
然后使用这些重写步骤来反向映射作用在这两个 $1$ 上的修改操作。
第一个 $1$ 并没有变化,所以原来的 $x$ 可以继续保留。
但第二个 $1$ 被替换成了 $2$,因此该宏调用需要被展开,$x$ 也需要被展开。
于是我们得到了最终的替换文本 $x+2$。  

%We first construct an unchanged expanded form, and then
% propagate back the changes along the unchanged expanded form. First,
% we replace the occurrences of parameters in the macro body by either
% the preprocessed argument or the unpreprocessed argument (both the original
% arguments without user changes). An occurrence of a parameter is
% replaced by a preprocessed argument if any of the following conditions is
% satisfied.
% \begin{itemize}
% \item The preprocessed argument starts with a left parenthesis.
% \item The preprocessed argument contains a top-level comma.
% \item The preprocessed argument contains any unclosed parenthesis.
% \item The preprocessed argument becomes different if we preprocess it again.
% \end{itemize}
% Otherwise, the occurrence is replaced by an unpreprocessed argument. In
% the previous example, $x$ does not satisfy any of the above condition,
% so $a$ in the body of $plus$ is replaced by the unpreprocessed
% argument, forming $x+x$.

% Next, 
% for any
% occurrence of a parameter replaced by an unpreprocessed argument, the
% rewriting steps of the argument is first copied at the new location to
% preprocess the argument, and then we copy

但是,我们不能总是把参数的重写步骤原封不动地拷贝到每个参数在展开式中出现的地方。
基本上说,当我们拷贝一个参数的重写步骤时,我们其实是在假设每一个参数
出现的位置可以用参数的未预处理形式来替换。而且替换后每个地方参数的
重写规则与它单独展开是一致的。
比方说,当展开 $plus(x)$ 时,我们可以把两个 $a$ 出现的地方替换成 $x$。
此时生成的展开式 $x+x$ 中两个参数出现的地方的展开重写规则序列和 $x$是一样的。
然后就能得到 $1+1$。
但是,情况并不总是这么理想,我们可以考虑以下的代码例子。

\begin{equation}\label{eqn:expandToParentheses}
  \begin{minipage}{0.8\columnwidth}
\begin{lstlisting}
#define p (x)
#define pplus(x) plus x hello
pplus(p)  
\end{lstlisting}
    \end{minipage}
\end{equation}

加上之前代码段 \ref{eqn:expansion} 中定义的宏,这里的例子最终会
预处理为$1+1\ hello$。但是,如果我们需要展开$pplus(p)$,
我们并不能把它直接展开成$plus\ p\ hello$,因为这只会展开成
$plus\ (1)\ hello$。这是因为在这里使用 $p$ 而不是 $(x)$ 
破坏了原有的宏调用。

%???????????

因此,我们需要在拷贝重写步骤的时候做一次安全检查。我们当且仅当
以下条件都不满足时,才会拷贝重写步骤。

\begin{itemize}
\item 预处理参数起始字符是一个左括号。
\item 预处理参数在顶层包含一个逗号。
\item 预处理参数中存在括号不匹配情况。
\item 如果我们重新预处理该参数,结果会不同。
\end{itemize}
前三个条件对应着之前例子中类似的情况:
预处理中参数在展开式中被用于其他的宏调用,替换它会导致展开式中的宏调用出错,
给程序引入错误。
% In the previous example, when $1+1$ is replaced by $1+2$, we expand
% the macro invocation $plus(x)$. Since the argument $x$ expands to
% $100$, the three conditions are satisfied and the body is replaced as
% $x+x$. Next we use the rewriting step of expanding $x$ to propagate
% the changes on the two $1$s. The first $1$ is not changed, so the
% original $x$ is kept. The second $1$ is changed into $2$, so the
% original macro invocation $x$ is also expanded, and we get the final
% text $x+2$, which is used to form a replacement of $plus$.
% To see an example where we should use the expanded argument, let us
% consider the following code.
% \begin{lstlisting}
% #define p (x)
% #define pplus(x) plus x hello
% pplus(p)  
% \end{lstlisting}
% With the definitions in code piece \ref{eqn:expansion}, the above code
% expands to $1+1 hello$. However, if we need to expand $pplus(p)$, we cannot
% expand it into $plus p hello$, because it will only expand to $plus
% (1) hello$. This is because the use of $p$ instead $(x)$ breaks the
% original macro invocation. In fact, all the first three conditions listed
% above are designed to prevent this case: the expanded macro is used as
% part of another macro invocation and replacing it with the unexpanded
% form may break the macro invocation.
最后一个条件对应着以下情况。
\begin{lstlisting}
#define id(x) x
id(plus p)
\end{lstlisting}
这段代码展开后得到$1+1$,但是如果我们把 $id(plus\ p)$
展开成  $plus\ p$,另外一个宏展开就会被阻塞。
换句话说,在宏调用中,一个参数会被扫描两次。
但是当宏调用被展开时,一个参数只会被扫描一次。
我们需要确认这之间的区别不会影响变换的正确性。

\subsection{正确性}\label{sec:correctness}
根据之前一节对反向变换的描述,我们可以简单推导出我们的反向变换满足
性质1、3、4和5。性质2将会在实验部分证明。
该反向变换满足性质1因为我们从不把修改操作映射到宏定义重。
该反向变换满足性质3因为如果我们通过修改操作引入了一个新的宏调用,
我们需要把一个普通的文本词变成宏调用的一部分。但是这个文本词本来
是通过规则4被预处理的。而规则4中反向变换的安全检查会阻止错误的发生。
该反向变换满足性质4因为我们只能在不得不展开某个宏调用时才会引入
新的修改操作,而该处没有修改操作,我们不会展开宏调用。
该反向变换满足性质5可以这样推导。
为了正式地证明性质5,我们需要诱导性地证明每一个重写步骤
只能要么保持同样行为,要么被一串等价的重写步骤
序列替换(该序列长度可以为0)。
所幸,当我们拷贝参数的重写步骤到展开式中时,
拷贝后的重写步骤可能会和之后的重写步骤混合起来。
我们可以指出,由于这些重写步骤序列作用位置不同且是覆盖性的,调换重写
步骤序列的位置并不会对结果产生影响。

\subsection{优化算法}\label{sec:optimization}
在现有的算法中,我们需要在每一步重写步骤后把整个程序的状态都记录下来,
即使只是很小的一部分发生了改变。
在反向变换中,每一步我们都需要调整所有修改操作的位移。
这样的算法使我们的系统十分笨重。

为了优化该算法,我们首先把源程序中的词序列切分成一组序列,
序列有自己独立的一套重写步骤。
这样,我们就能把每个序列当作单独的程序来做反向变换,再把修改融合到一起。
给定一个重写步骤 $(src, i, ctx, r)$,如果在 $i$ 位置的词不是由
之前的重写规则生成的,也就是说在原来的代码中就有,那么我们就可以认为
位置 $i$ 可以是一个\emph{切分点}。
有了切分点的定义后, 由于词是程序的最小单位,我们可以认为没有
重写规则会同时作用在切分点的两边。
我们依据切分点把程序切分成序列的集合,然后依次在序列上做反向变换,再融合这些修改。
当然,再合并两个子序列时,我们还需要特殊检查两边程序在预处理后,
合并起来不会形成新的宏调用。这个检查方法与之前提到的规则4的安全检查方法类似。

比方说,之前的代码段~\ref{eqn:smallexample}可以被切分成三个子序列:
宏定义, $hello$,和 $x$。
假设之前定义过宏 $plus$。 如果程序编辑工具把 $hello$ 修改成了 $plus$,
把 $100$ 修改成了 $(x)$, 那么两个子序列在预处理后,合并起来就在边界上
形成了一个新的宏调用。我们应该报错。


\section{扩展到完整的C预处理器}\label{sec:fullC}
本节中我们讨论如何把我们的算法扩展到支持完整的C预处理。
由于篇幅限制,我们只讨论核心思想而非细节。

为了在正向预编译中支持 \# 和 \#\#操作,我们需要在之前提到的规则3的
正向变换中添加两个新的小步骤。一个为了支持\#可以字符化一个词,一个支持
\#\#可以连接两个词。
另外,如果在参数中出现这两类操作,我们会直接使用未预处理的形式
来替换展开式中出现的位置。

在反向变换中,我们需要添加3个新的扩展。第一,我们需要为原有的规则3
依据之前所添加的小步骤,设计两个新的反向步骤来支持\#和\#\#操作。
第二,我们需要检查使用这两个操作的代码在预处理之前和预处理之后是否
有同样的修改。最后,当展开一个宏的时候,我们并不需要恢复\#和\#\#
操作,因为他们不能出现在没有宏定义指令的地方。

为了在正向变换中支持~\code{\#include}指令,我们需要为~\code{\#include}指令
添加新的一条重写规则。其内容为删除这行宏指令并引入新的文件内容。
在反向变换中,我们需要记录加入了哪些文件和内容,
并根据不同的~\code{\#include}指令,把这些文件引入都一一返回回去,
这样才能保持一致。

最后,一个完整的C预处理器不允许宏的循环调用。
为了支持这个功能,我们需要在正向重写规则中添加一些细节的考察。
当我们在不断重写由宏 $m$ 展开的词时,我们需要检查是否有再次使用这个宏展开的情况。
如果出现,应该报错。

\section{扩展到其他修改操作上}\label{sec:extend-other-changes}
在上文中我们详细地讨论了如何处理修改操作。
现在让我们来讨论如何处理插入和拷贝操作。
我们可以注意到,插入操作可以直接被转换为修改操作。
如果我们要在词 $x$ 之前插入一个词 $y$, 那我们可以把这个插入操作
转换为一个将 $x$ 变为  $y\ x$ 的替换操作。
但是,这样的转换有时会引入不必要的宏调用展开。
比方说,如果我门在之前的代码段~\ref{eqn:smallexample}中
在 $100$ 之前插入了一个词 $y$,那么我们在反向变换中并不需要拆开 $x$ 这个宏。
但如果我们把这个插入操作转换成一个替换操作的话,比如把 $100$ 变成 $y\ 100$,
那么我们就需要把宏 $x$ 拆开。

刚才的例子是在切分点上插入了一段词,而这样的插入可以简单证明,不会影响到宏,
也不需要再反向编译时展开宏。
为了减少不必要的反向宏展开,我们把插入在切分点的修改操作
看作是添加了一个独立的子序列,然后直接在上面检查是否可以应用规则4。
如果其他规则可以在这个插入序列上应用,那说明这段插入是错误的,将会由我们的安全检查报错。
这样一来,其余的代码依然可以使用之前的算法处理。

拷贝操作的处理和插入操作类似。二者之间唯一的区别是,拷贝操作引入的代码段中
可能包含宏调用,而我们需要尽量不展这些宏调用。
为此,我们需要对这段词序列做一次反向转换。
这次反向转换算法如下:
(1)首先,我们假定对这段代码以外的所有预处理后代码做删除操作。
(2)然后,我们剔除掉那些没有在目标代码段定义的宏并做一次反向变换。
这样我们就能保证只有在目标代码段的上下文环境中定义的宏会被正确地反向变换回去。
最后我们可以把已经做好反向变换的这段代码插入到预处理前的相应位置中。

% Instead, we deal with an insertion by concatenating three
% token sequences: the tokens before the insertion, the inserted tokens,
% and the tokens after the insertion. We first generate a change that
% deletes all tokens before the insertion, and perform a backward
% transformation to get the token sequence before insertion.
% In the above example, we first generates a change to delete $100$, and
% a backward transformation will give us 


% \section{Approach Overview}

% There are three key issues.
% \begin{itemize}
% \item We show that different preprocessor directives and macro
%   operators can be captured by a unified concept, text unit. The
%   execution of the preprocess can be captured by the composition of
%   text units, which enables a structural decomposition of the backward transformation.
% \item Even with the composition structure, we still need to deal with
%   the intersection of each type of text unit and each change
%   operation. We further show that there exists a core change
%   operation, replacement. We only need to consider the propagation of
%   replacement on each type of text unit. The other change operations
%   can be easily mapped to the core operation.
% \item Since our bidirectional transformation not only changes the
%   values transformed by the transformation but also the transformation
%   program itself, it will produce two outputs, one is the changed
%   program and one is the change to the data. When we apply the change
%   to the data to the program, we get the final result.
% \end{itemize}


% \newcommand{\replace}[2]{\ensuremath{[{#1}]\backslash {#2}}}
% \newcommand{\copyOpr}[3]{\ensuremath{[{#1},{#2}]\rightarrow [{#3}]}}
% \newcommand{\del}[2]{\ensuremath{\cancel{[{#1},{#2}]}}}
% \section{Modeling the Changes}

% A location or offset of a character in the source code file is 
% A location in the source code file is a pair $(f, o)$, where $f$ is
% the name of the source file, and $o$ is the offset from the start of
% the file, but treating all continuous whitespace characters as one character.

% A replacement is a pair $(l, s)$, where $l$ is a location on which the
% character is not a whitespace, and $s$ is
% a string that will replace the character at $l$.

% An insertion is a pair $(l, s)$, where $l$ is a location on which the
% character is a whitespace, and $s$ is a string that will be inserted
% at $l$.

% A copy is a pair $(l_t, l_b, l_e)$, showing the string between $l_b$
% and $l_e$ is copied into the location $l_t$, where $l$ is a location
% on which the character is a whitespace. 





% % Three atomic operations: replace, insert, copy, and delete.

% % Replace: a tuple $(p, s)$, denoted as $\replace{p}{s}$, showing the char at $p$ is replaced by $s$.

% % Copy: a tuple $(b, e, p)$, denoted as $\copyOpr{b}{e}{p}$, showing the string between $b$ and $e$ is
% % copied to replace the character at $p$.

% % Delete: a tuple $(b, e)$, denoted as $\del{b}{e}$, showing the string between $b$ and $e$ is
% % deleted.

% % Conflict rules: copy may overlap with the other two operations, but
% % the copied content is not modified. 

% % A set of standard operations: replace, copy, move, and delete.

% % RReplace: a tuple $(b, e, s)$, where the string between $b$ and $e$ is
% % replaced by $s$. RReplace can be modelled by Replace $(b, s)$ and
% % delete $(b, e+1)$.

% % CCopy is equal to copy.

% % DDelete is equal to delete.

% % MMove is a tuple $(b, e, p)$ that is equal to copy $(b, e, p)$ and
% % delete $(b, e)$.

% % \begin{definition}[Correctness]
  
% % \end{definition}


% % \begin{definition}[Completeness]
  
% % \end{definition}

% \section{Modeling Macro Expansion}
% % confluence in backward transformation? -- a sub case of correctness

% A text segment is a sequence of text units. A text unit is one of the
% following types: 




% \section{Forward Transformation}

% \section{Backward Transformation}

% We need to perform a check of the name collision

%%% Local Variables: 
%%% mode: latex
%%% TeX-master: "main"
%%% End: 

	
\section{Evaluation}
\label{sec:evaluation}
\subsection{Research Questions}
In this section we focus on the following research questions.
\begin{itemize}
\item {\bf RQ1: Macro Preservation. } According to requirement 2, 
  our approach aims to preserve existing macro invocations. How does
  the strategy perform on actual programs? How does it compare to
  other techniques?
\item {\bf RQ2: Correctness. } Our approach is guaranteed to be
  correct according to requirements 4 and 5. How important is this
  correctness? How does our approach compare to other techniques that
  do not ensure correctness?
\item {\bf RQ3: Failures. }  Our approach may report a failure when it
  cannot find a proper way to propagate the change. How often does
  this happen? Are the failures false alarms (there exists a suitable
  change but our approach cannot find it)?
\end{itemize}

To answer these questions, we conducted a controlled experiment to
compare our approach with the two naive approaches described in
\secref{sec:naive} on a set of generated changes on Linux kernel
source code. In the rest of the section we describe the details of the
experiment.

\subsection{Setup}

\subsubsection{Implementation}
We have implemented our approach in Java by modifying
JCPP\footnote{\url{http://www.anarres.org/projects/jcpp/}}, an open
source C Preprocessor. We also implemented the two naive approaches in
\secref{sec:naive} for comparison. Our implementation and experimental data
can be found on our web
site\footnote{\url{https://github.com/harouwu/BXCPP}}.

\subsubsection{Benchmark}
Our experiment was conducted on the Linux kernel version 3.19. We chose
Linux source code because Linux kernel is one of the most widely used
software projects implemented in C. It contains contributions from many
developers, and has a lot of preprocessor directives and macro
invocations.

To conduct our experiment, we need a set of changes on the Linux
kernel code. Since we concern about how different backward
transformations affect preprocessing, we generated changes only in functions that
contain macro invocations. To do this, we first randomly selected 180 macros
definitions from the kernel code. Since there are far more object-like
macros than function-like macros, we
would select very few function-like macros if we use pure random selection. So we controlled the ratio
between object-like and function-like macros to be 1.5 : 1. Based on
the selected macros, we randomly selected a set of functions which contain
invocations to the macros. Finally, we randomly selected 8000 lines from
the functions. There are in total 133 macro invocations in the
selected lines.

Next we generated a set of changes on the selected lines. To simulate real
world changes, we randomly generated two types of changes. The first
type is token-level change, in which we randomly replace/delete/insert
a token. The second type is statement-level change, in which we delete
a statement or copy another statement to the current location. These
two types of changes are summarized from popular bug-fixing
approaches~\cite{le2012genprog,QiMLDW14,kim2013automatic}. The statement-level changes are directly used by GenProg~\cite{le2012genprog} and
RSRepair\cite{QiMLDW14}. The token-level changes simulate small changes such as
replacing the argument of a method or change an operators used in
approaches such as PAR~\cite{kim2013automatic}.

More concretely, we had a probability $p$ to perform an operation on
each token, where the operation is one of insertion, replacement and
deletion, which had equal probability. The replacement was performed
by randomly mutating some characters in the token. The insertion was
performed by randomly copying a token from somewhere else. Similarly,
we had a probability $q$ to perform an operation on each statement,
where the operation is copy or deletion. The copied statement was
directly obtained from the previous statement. We recognized a
statement by semicolon.

Different tools may have different editing patterns:
a migration tool typically changes many places in a program, whereas a
bug-fixing tool may change a few places to fix a bug. To simulate these two different
densities of changes, we used two different set of probabilities. For
the high-density changes, we set $p=0.33$ and $q=0.1$. For
the low-density changes, we set $p=0.1$ and $q=0.05$.

We generated ten sets of changes, five with high-density and five with
low-density. The number of the changes generated for each set is shown in Table~\ref{tbl:changes}.
\begin{table}[htbp]
\caption{Changes generated for the experiment}\label{tbl:changes}
\centering
% \begin{tabular}{|l|lllll|lllll|}
%   \hline
%   Density & \multicolumn{5}{c|}{Low} & \multicolumn{5}{c|}{High} \\
%   \hline
%   Set & 1 & 2& 3& 4 & 5 & 6 & 7 &8 &9 &10\\
%   \hline
%   Changes & 952 & 885 & 956 & 967 & 884 & 3133 & 3136 & 3088 & 3123 &
%                                                                       3048 \\
% \hline
% \end{tabular}
\begin{tabular}{|l|l|lllll|}
  \hline
  \multirow{2}{1cm}{Low Density} & Set & 1 & 2 & 3 & 4 & 5  \\
  \cline{2-7}
                                 & Changes & 952 & 885 & 956 & 967 & 884 \\
  \hline
  \multirow{2}{1cm}{High Density} & Set & 6 & 7 & 8 & 9 & 10 \\
  \cline{2-7}
                                 & Changes & 3133 & 3136 & 3088 & 3123 & 3048\\
  \hline
\end{tabular}
\end{table}

\subsubsection{Independent variables}
We considered the following independent variables. (1)
\emph{Techniques}, we compared our approach with the two naive
solutions, per-file and per-line. (2) \emph{Density of changes}, we
evaluated both on the five high-density change sets and the five
low-density change sets. % (3) \emph{Types of changes}, we distinguished token-level changes, statement-level changes, and
% combinations of them.

\subsubsection{Dependent variables}
We considered two dependent variables. (1) \emph{Number of remaining
  macro invocations}. We re-ran the preprocessor after the backward
transformation, and counted how macro invocations are expanded during
preprocessing. Since none of the techniques will actively introduce
new macro invocations% \footnote{Strictly speaking, new macro
  % invocations may be introduced passively by the user, e.g., a token
  % is accidentally changed into an object-like macro. Since the
  % probability is very small, we can safely ignore these cases.}
, the
number of expanded invocations is the number of remaining invocations.
To avoid noise from included files, we count only the macro
invocations in the current file.
(2) \emph{Number of errors}. We re-ran the preprocessor, and compared
the new preprocessed program with the previously changed program by
Unix file-comparing tool $fc$. Every time $fc$ reported a difference,
we counted it as an error. (3) \emph{Failures}. Our approach may fail
to propagate the changes, and we record whether a failure is reported
for each change set.

\subsection{Threats to Validity}
A threat to external validity is whether the results on generated
changes can be generalized to real world changes. To alleviate this
threat, we used different types of changes and different density of
changes, in the hope of covering a good variety of real-world changes.

A threat to internal validity is that our implementation of the three
approaches may be wrong. To alleviate this threat, we investigated all
errors we found in the experiments, to make sure it is a true defect
of the respective approach but not a defect in our implementation.

\subsection{Results}
\begin{table}[htbp]
  \caption{Experimental Results}\label{tbl:results}
\centering
\begin{tabular}{|l|l|lllll|}
  \hline
  Low Density & Set & 1 & 2 & 3 & 4 & 5\\
  \hline
  \multirow{3}{*}{Our Approach} &  Macros & 73 & 75 & 72 & 80 & 81 \\
  \cline{2-7}
              &Errors & 0 & 0 & 0 & 0 & 0  \\
  \cline{2-7}
              & Failures & n & n & n & n & n \\
  \hline
  \multirow{3}{*}{Per-Line} & Macros & 23 & 25 & 23 & 20 & 26 \\
  \cline{2-7}
              & Errors & 6 & 7 & 6 & 7 & 7  \\
  \hline
  \multirow{3}{*}{Per-File} & Macros & 0 & 0 & 0 & 0 & 0  \\
  \cline{2-7}
              & Errors & 0 & 0 & 0 & 0 & 0 \\
  \hline
  \hline
  High Density & Set & 6 & 7 & 8& 9& 10\\
  \hline
  \multirow{3}{*}{Our Approach} &Macros & 47 & 51 & 53 & 48 & 44 \\
  \cline{2-7}
              &  Errors & 0 & 0 & 0 & 0 & 0  \\
  \cline{2-7}
              & Failures & n & n & n & n & n \\
  \hline
  \multirow{3}{*}{Per-Line} & Macros & 9 & 7 & 7 & 8 & 10  \\
  \cline{2-7}
              & Errors & 6 & 6 & 7 & 6 & 6 \\
  \hline
  \multirow{3}{*}{Per-File} & Macros & 0 & 0 & 0 & 0 & 0  \\
  \cline{2-7}
              & Errors & 0 & 0 & 0 & 0 & 0 \\
  \hline\end{tabular}
\\
\parbox{\columnwidth}{\footnotesize Row ``Macros'' shows the number of
    remaining macros. Row ``Errors'' shows the number of errors caused. Row ``Failures'' indicates whether a
    failure is reported in the backward transformation.}
\end{table}
% \begin{table*}[htbp]
% \caption{Macros remainings and errors of different algorithms and data set}
% \centering
% \begin{tabular}{l|cc|cc|cc}
% \hline
% Algorithm  &High Mutation & &Low Mutation & &Average &  \\
%  &Remain &Error &Remain &Error &Remain &Error \\
% \hline
% CPP-TRANS &120 &0 &243.8 &0 &181.9 &0 \\
% PER-FILE &0 &0 &0 &0 &0 &0  \\
% PER-LINE &25.6 &5.8 &64.2 &2.4 &44.9 &4.1 \\
% \hline
% \end{tabular}
% \end{table*}

The result of our evaluation is shown in Table~\ref{tbl:results}. We
discuss the results with respect to the research questions below.

\subsubsection{RQ1} As we can see, our approach preserves macro invocations. Per-line preserves very few macro
invocations, while per-file, as we expected, preserves no macro
invocations. We further investigated why per-line preserves so few
macro invocations. One main reason we found is that there are usually
multiple macro invocations per line, and per-line will expand all of
them if any tokens in this line is changed.
\subsubsection{RQ2} Our approach and per-file lead to no errors while
several errors are caused by per-line. This is because there are quite
a few macro invocations that cross multiple lines. These macros 
take expressions or statements as argument, which are usually too long
to be included in one line.
\subsubsection{RQ3} As discussed before, our approach may report a failure during the
backward transformation. This is usually because the changes
accidentally introduce a new macro invocation in the preprocessed code, where there is no way
to satisfy PUTGET. 
However, we do not
observe any such cases in our experiment. The reason is that
macros usually have special names and it is not easy to collide with a
macro name by copying or mutation. Note the other two
approaches never report a failure, so the corresponding fields in Table~\ref{tbl:results} are left blank. 

Although probably being rare in practice, theoretically our approach
may report false alarms: our approach reports a failure but
a correct change on the source program exists. {For example, let consider the
  following code piece,
\begin{lstlisting}
#define p (x)
plus p
\end{lstlisting}
where $plus$ is the macro defined in code
piece~\eqref{eqn:expansion}. After preprocessing, this code piece becomes
$plus\ (x)$. If we change the last parenthesis into $)\ hello$, our
approach reports a failure because first $p$ will be expanded and then
the expanded content forms a new macro invocation with $plus$.
However, there exists a feasible change: replacing $p$ with $hello\ p$.}

%%% Local Variables:
%%% mode: latex
%%% TeX-master: "main"
%%% End:

	\chapter{相关工作}
\label{sec:related}

\section{双向变换领域}
我们工作的灵感来自于双向变换领域的研究。
双向变换领域中一个经典的使用情况就是数据库设计中的
\emph{view-update problem}~\parencite{BaSp81,DaBe82,Hegner90,Cui2000,Fegaras2010}:
一个域表示一个为一条查询指令计算好的数据库,
研究工作主要关注于如何把在这个域上做的修改操作反向升级到源数据库中。
这个工作就像\emph{模块转化}在软件进化中一样重要~\parencite{YuLHHKM12,XiLHZTM07}。

同时,针对这种双向变换的应用,学术界也设计了许多语言使得工作更加方便。
其中广为人知晓的是 \emph{透镜}理论框架~\parencite{HuMT04,MuHT04aplas,Foster:2007,BoFPPS08,FoPP08,WaGMH10,Diskin2011,Hofmann2012,FoMV12,RaLFC13}。
在这套框架下涵盖了许多为双向变换提供连接因子的语言架构。
另一种比较主流的思想是自动化地为现存的非双向变换的程序
找到对应的反向程序。这类的研究称为 \emph{双向变换化(Bidirectionalization)}
~\parencite{MaHNHT07,Voigtlander09bff,voigtlander2010combining,WaGW11,VoHMW13,WaGMH13,MaWa13,MaWa13ppdp,WaNa14,MaWa14}。
在软件工程的模块转换领域,需要被转换的数据通常以图(并非是树)的结构来保存。
此时主流会使用一个关系型,而非函数型,的方法来描述不同模块之间的双向的不同的关系
~\parencite{qvt,Stevens2010,Schurr1994,Schurr2008,HiHIKMN10,Hidaka:2011}。
但是,我们工作的要求现在并没有一种双向变换的技术可以满足。这是因为在我们的模型中,
不仅作为数据的代码会出现变化,作为转换程序本身的宏、预处理指令,也会有相应的变化。
因此,我们也为这类数据和转换程序都会变化的双向变换模型,提出了自己见解。


\section{分析和修改未预处理的C代码}
C的预处理器给静态程序分析带来了巨大的挑战。
由于C的预处理器支持预处理变量,导致分析时,场景数量组合型快速增长。
这使得传统的每次只处理一个变量的程序分析方法变得不可行。
也仅直到最近,通过\emph{family-based
analyses}~\parencite{Kastner2011,Gazzillo2012,Liebig2013}
的方法才能对未预处理的C代码进行可靠的解析与分析。
之前的许多工具都时常出现不可靠的操作,
或者只能限制在非常严格的使用情况下使用~\parencite{Baxter2001,Garrido2005,Padioleau2009}。

类似的,为了处理这种多变量的情况,在C代码重构的研究中学术界也花了很多功夫。
大多数方法~\parencite{Garrido2002,Vittek2003,Spinellis2003,Garrido2013}都试图
找到一个能够同时表示C语法和预处理器指令的合适模型。
最近有一项工作~\parencite{Overbey2014}提出了另外一种方法:
为某一个变量做重构,但如果这对其他变量产生影响就阻止重构。
这项工作是基于预处理变量修改相互之间往往影响很小的观测而设计出来的。
我们可以看到,解决预处理前代码的分析现在并没有优秀的解决方案。

需要承认的是,我们的项目现在暂时也只考虑一个变量。
在未来我们可能会把我们的工作和多变量的算法相结合。
但是,即使只能处理一个变量,我们的项目在许多方面都十分有用:
(1)许多实际工作中的程序,尽管有许多条件编译选项,没有许多预处理变量。
(2)Overbey等人~\parencite{Overbey2014}在他们的论文中指出,
现实中修改一个变量往往不会对其他变量造成影响。
%  However, our approach requires minimal
% changes to existing tools, so can potentially be applied to many areas
% that do not require specialized effort.

% Refactoring unpreprocessed C code is extremely difficult. Even without considering the above mentioned problem with parsing, dealing with macro expansion while editing remains tricky~\cite{Garrido2002,Vittek2003,Spinellis2003,Garrido2013,Overbey2014}. Even for relatively simple tasks such as identifying and removing dead code sophisticated analyses are required~\cite{Baxter2001,Tartler2011}. 

% Our proposal in this paper does not apply to the cases here. We do
% not consider multiple configurations, and we do not allow explicit
% modification of preprocessor directives.

% Though our approach is able to handle some refactorings, our approach
% cannot replace these approaches because (1) as discussed before, our
% approach do not consider multiple configurations, and (2) our approach
% does not change preprocessor directives, which is necessary in
% refactorings such as renaming macros and extract macros.

%Work by Christian Kastner
%
%Parsing C/C++ code without pre-processing by Y Padioleau, CC'09
%
%SuperC: parsing all of C by taming the preprocessor, PLDI'12

\section{C预处理器上的实例调查}
许多年来,学术界不乏各类对C预处理器十分有见解的学术实例调查工作
~\parencite{Spencer92,ernst2002empirical,Liebig2011}。
同时也有一些C预处理的替代品被提出,比如语法预处理器~\parencite{Weise1993,McCloskey:2005}
或者面向侧面开发方法~\parencite{Lohmann2006,Adams2009,Boucher2010}。
但是直至今日,业界并没有什么采用这类替代预处理器的迹象,
C预处理器依然是主流的工具。
这也说明C预处理器带来的麻烦依然广泛存在。



%%% Local Variables: 
%%% mode: latex
%%% TeX-master: "main"
%%% End: 


	% 结论。
	\chapter{总结}
\section{Conclusion}
\label{sec:conclusion}

为程序编辑工具处理C预处理器是十分困难的事情。
因此,所以许多工具不是只能给出不可靠的结果或者完全放弃处理预处理器。
本文中我们使用双向变换的方法把预处理器带来的麻烦从
程序编辑工具中分离出来单独解决。
这样程序编辑工具就能专注于预处理之后的代码,达到更模块化的设计效果。

C预处理器涉及到的问题也可以看作是转换系统问题中特殊的一类。
在这类问题中,转换程序和数据可能同时被转换。
这类问题还有类似双向变换的PHP~\parencite{wang2012automating}。
现有的方法~\parencite{wang2012automating}往往使用十分复杂且专门的方法并陷入双向变换
性质的定义和正确性的讨论中去。
本文所提出的方法对这一类的问题都有适用价值:该方法把转换程序当作数据,
并在更高一层的操作语义上双向变换。
我们也希望未来能看到一个通用的理论能够完美解决这一类问题。



%%% Local Variables: 
%%% mode: latex
%%% TeX-master: "main"
%%% End: 


	% 正文中的附录部分。
	\appendix
	% 排版参考文献列表。
	\printbibliography[
		% 使“参考文献”出现在目录中;如果同时要使参考文献列表参与章节编号,
		% 可将“bibintoc”改为“bibnumbered”。
		heading = bibintoc,
		% 单独设定排序方案。此设定会局部覆盖之前的全局设置。
		% 注:只有同时使用 2.x 或之后版本的 biblatex 和相应兼容版本的 biber,
		% 才能对每个 \printbibliography 命令采用不同的排序方案,
		% 否则只能在载入 biblatex 宏包时就(全局)指定排序方案。
		% 在这样的情况下,请去掉所有的 sorting 选项,否则可能出错。
		% sorting = ecnty
	]

	% 以下为正文之后的部分,默认不进行章节编号。
	\backmatter
	% 致谢。
	% vim:ts=4:sw=4
% Copyright (c) 2014 Casper Ti. Vector
% Public domain.

\chapter{致谢}

毕业论文写到这里告一段落,在此感谢大学四年来在学业上对我有过帮助的老师,亲友。

首先一定要感谢的是熊英飞老师。
熊英飞老师在我大二那年来到北大,我和他在计算机著名神课《计算机系统导论》上相识。
之后我加入熊老师的课题组,一直到今天,已经快2年半。
熊老师平日里一直是组里/所里的老好人,教授们喊他小熊,学生们喊他熊神。
他和学生完全打成一片,有教授的学识却无教授的架子。
也正是因为他和蔼,对学生容忍度很高,拖沓的我总是能钻空子,科研的时间虽长,但论文成果屈指可数。

然而在熊老师组内的日子并不是纯粹的游手好闲,每次去理科一号楼1431找熊老师时,总能看到他
或是在电脑前写论文、或是和同学讨论、或是对着白板思考问题。每当看到这一幕时,
我心中总有一种惭愧感。当然熊老师是不知道的,他总是会放下手里的事情接待我。
在这些找他的谈话中,作出了我大学生活中几个关键的决定:实习还是科研?选择哪所研究生院的录取?

我记得有一次闲聊时,听熊老师聊过当年他当年做学生时,有的老师完全不顾学生。
但当他遇到东京大学的胡振江教授后,他发现好的老师应该去感化学生,也因此立下志向做一名老师。
教学的理论我完全不了解,但作为学生,我大学四年真正地很幸运地被熊老师感化过。
我想,熊老师的确做到了他心中理想的教师的样子。
我为自己感到幸运,也为他感到高兴。

接下来感谢的名单不分先后。

感谢学校、学院对我的培养。大学四年中北大、信科给了我学术、工作上的舞台。我三次出国交流,
担任过学生干部,拿过“五四”最高荣誉等等。这中间离不开学校的栽培和包括梅宏院士、陆俊林老师、马郓学长等
学院老师前辈的鼓励和指导。

感谢黄震、赵明民、黄元三位室友。能和优秀的人相伴是我的荣幸。
你们三个总是霸占着年级前十和各种奖学金,赵明民还在本科期间就在MobiCom上发表文章。
这些光环总是鞭策着游业散漫的我前进。
前三年大家都太忙,感情尚可。但到了大四,四个人反而更坦诚相待,变得心心相惜起来。
兄弟这个词一般不适合我们四个从南方来的男生,
但我心里知道一生的友谊尽在不言中。
今年秋天我们四个天各一方,望时光不老,我们不散。

感谢我的爸爸妈妈,大学四年总没什么时间回家。谢谢你们对我一直的支持和关心。大四的暑假决定哪里也不去,留在家里陪爸妈。

感谢大学间认识的好友们,你们一直是我内心力量的来源。

感谢北京这座城市,接纳了一个讨厌又鄙视她的上海人四年。
感谢她提供了每一处地点都与我的一些故事相联系。
到头来,讨厌她的人也开始留恋。
我想这是包容。

至此致谢结束。我的大学生活即将画上句号。
希望未来的人生我能做对选择,
变成我想成为的那个自己。





	% 此后不排版页眉或页脚。
	\cleardoublepage
	\pagestyle{empty}

	% 原创性声明和使用授权说明。
	% vim:ts=4:sw=4
%
% Copyright (c) 2008-2009 solvethis
% Copyright (c) 2010-2015 Casper Ti. Vector
% All rights reserved.
%
% Redistribution and use in source and binary forms, with or without
% modification, are permitted provided that the following conditions are
% met:
%
% * Redistributions of source code must retain the above copyright notice,
%   this list of conditions and the following disclaimer.
% * Redistributions in binary form must reproduce the above copyright
%   notice, this list of conditions and the following disclaimer in the
%   documentation and/or other materials provided with the distribution.
% * Neither the name of Peking University nor the names of its contributors
%   may be used to endorse or promote products derived from this software
%   without specific prior written permission.
%
% THIS SOFTWARE IS PROVIDED BY THE COPYRIGHT HOLDERS AND CONTRIBUTORS "AS
% IS" AND ANY EXPRESS OR IMPLIED WARRANTIES, INCLUDING, BUT NOT LIMITED TO,
% THE IMPLIED WARRANTIES OF MERCHANTABILITY AND FITNESS FOR A PARTICULAR
% PURPOSE ARE DISCLAIMED. IN NO EVENT SHALL THE COPYRIGHT HOLDER OR
% CONTRIBUTORS BE LIABLE FOR ANY DIRECT, INDIRECT, INCIDENTAL, SPECIAL,
% EXEMPLARY, OR CONSEQUENTIAL DAMAGES (INCLUDING, BUT NOT LIMITED TO,
% PROCUREMENT OF SUBSTITUTE GOODS OR SERVICES; LOSS OF USE, DATA, OR
% PROFITS; OR BUSINESS INTERRUPTION) HOWEVER CAUSED AND ON ANY THEORY OF
% LIABILITY, WHETHER IN CONTRACT, STRICT LIABILITY, OR TORT (INCLUDING
% NEGLIGENCE OR OTHERWISE) ARISING IN ANY WAY OUT OF THE USE OF THIS
% SOFTWARE, EVEN IF ADVISED OF THE POSSIBILITY OF SUCH DAMAGE.

{
	\vspace*{\fill}
	\centerline{\bfseries\zihao{-2}北京大学学位论文原创性声明和使用授权说明}

	\vskip 4em
	\centerline{\bfseries\zihao{-3}原创性声明}
	\vskip 1em

	本人郑重声明:
	所呈交的学位论文,是本人在导师的指导下,独立进行研究工作所取得的成果。
	除文中已经注明引用的内容外,
	本论文不含任何其他个人或集体已经发表或撰写过的作品或成果。
	对本文的研究做出重要贡献的个人和集体,均已在文中以明确方式标明。
	本声明的法律结果由本人承担。
	\vskip 1em
	\rightline{%
		论文作者签名:\hspace{5em}%
		日期:\hspace{2em}年\hspace{2em}月\hspace{2em}日%
	}

	\vskip 4em
	\centerline{\bfseries\zihao{-3}学位论文使用授权说明}
	\centerline{\zihao{5}(必须装订在提交学校图书馆的印刷本)}
	\vskip 1em

	本人完全了解北京大学关于收集、保存、使用学位论文的规定,即:
	\begin{itemize}
		\item 按照学校要求提交学位论文的印刷本和电子版本;
		\item 学校有权保存学位论文的印刷本和电子版,
			并提供目录检索与阅览服务,在校园网上提供服务;
		\item 学校可以采用影印、缩印、数字化或其它复制手段保存论文;
		\item 因某种特殊原因需要延迟发布学位论文电子版,
			授权学校在 $\square$\nobreakspace{}一年 / %
			$\square$\nobreakspace{}两年 / %
			$\square$\nobreakspace{}三年以后在校园网上全文发布。
	\end{itemize}
	\centerline{(保密论文在解密后遵守此规定)}
	\vskip 1em
	\rightline{%
		论文作者签名:\hspace{5em}导师签名:\hspace{5em}%
		日期:\hspace{2em}年\hspace{2em}月\hspace{2em}日%
	}

	% 若需排版二维码,请将二维码图片重命名为“barcode”,
	% 转为合适的图片格式,并放在当前目录下,然后去掉下面 2 行的注释。
	%\vskip 4em \noindent
	%\includegraphics[height = 5em]{barcode}

	\vspace*{\fill}\par
}


\end{document}

