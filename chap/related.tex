\chapter{相关工作}
\section{Related Work}
\label{sec:related}

\subsection{双向变换领域}
我们工作的灵感来自于双向变换领域的研究。
双向变换领域中一个经典的使用情况就是数据库设计中的
\emph{view-update problem}~\parencite{BaSp81,DaBe82,Hegner90,Cui2000,Fegaras2010}:
一个域表示一个为一条查询指令计算好的数据库,
研究工作主要关注于如何把在这个域上做的修改操作反向升级到源数据库中。
这个工作就像\emph{模块转化}在软件进化中一样重要~\parencite{YuLHHKM12,XiLHZTM07}。

同时,针对这种双向变换的应用,学术界也设计了许多语言使得工作更加方便。
其中广为人知晓的是 \emph{透镜}理论框架~\parencite{HuMT04,MuHT04aplas,Foster:2007,BoFPPS08,FoPP08,WaGMH10,Diskin2011,Hofmann2012,FoMV12,RaLFC13}。
在这套框架下涵盖了许多为双向变换提供连接因子的语言架构。
另一种比较主流的思想是自动化地为现存的非双向变换的程序
找到对应的反向程序。这类的研究称为 \emph{双向变换化(Bidirectionalization)}
~\parencite{MaHNHT07,Voigtlander09bff,voigtlander2010combining,WaGW11,VoHMW13,WaGMH13,MaWa13,MaWa13ppdp,WaNa14,MaWa14}。
在软件工程的模块转换领域,需要被转换的数据通常以图(并非是树)的结构来保存。
此时主流会使用一个关系型,而非函数型,的方法来描述不同模块之间的双向的不同的关系
~\parencite{qvt,Stevens2010,Schurr1994,Schurr2008,HiHIKMN10,Hidaka:2011}。
但是,我们工作的要求现在并没有一种双向变换的技术可以满足。这是因为在我们的模型中,
不仅作为数据的代码会出现变化,作为转换程序本身的宏、预处理指令,也会有相应的变化。
因此,我们也为这类数据和转换程序都会变化的双向变换模型,提出了自己见解。


\subsection{分析和修改未预处理的C代码}
C的预处理器给静态程序分析带来了巨大的挑战。
由于C的预处理器支持预处理变量,导致分析时,场景数量组合型快速增长。
这使得传统的每次只处理一个变量的程序分析方法变得不可行。
也仅直到最近,通过\emph{family-based
analyses}~\parencite{Kastner2011,Gazzillo2012,Liebig2013}
的方法才能对未预处理的C代码进行可靠的解析与分析。
之前的许多工具都时常出现不可靠的操作,
或者只能限制在非常严格的使用情况下使用~\parencite{Baxter2001,Garrido2005,Padioleau2009}。

类似的,为了处理这种多变量的情况,在C代码重构的研究中学术界也花了很多功夫。
大多数方法~\parencite{Garrido2002,Vittek2003,Spinellis2003,Garrido2013}都试图
找到一个能够同时表示C语法和预处理器指令的合适模型。
最近有一项工作~\parencite{Overbey2014}提出了另外一种方法:
为某一个变量做重构,但如果这对其他变量产生影响就阻止重构。
这项工作是基于预处理变量修改相互之间往往影响很小的观测而设计出来的。
我们可以看到,解决预处理前代码的分析现在并没有优秀的解决方案。

需要承认的是,我们的项目现在暂时也只考虑一个变量。
在未来我们可能会把我们的工作和多变量的算法相结合。
但是,即使只能处理一个变量,我们的项目在许多方面都十分有用:
(1)许多实际工作中的程序,尽管有许多条件编译选项,没有许多预处理变量。
(2)Overbey等人~\parencite{Overbey2014}在他们的论文中指出,
现实中修改一个变量往往不会对其他变量造成影响。
%  However, our approach requires minimal
% changes to existing tools, so can potentially be applied to many areas
% that do not require specialized effort.

% Refactoring unpreprocessed C code is extremely difficult. Even without considering the above mentioned problem with parsing, dealing with macro expansion while editing remains tricky~\cite{Garrido2002,Vittek2003,Spinellis2003,Garrido2013,Overbey2014}. Even for relatively simple tasks such as identifying and removing dead code sophisticated analyses are required~\cite{Baxter2001,Tartler2011}. 

% Our proposal in this paper does not apply to the cases here. We do
% not consider multiple configurations, and we do not allow explicit
% modification of preprocessor directives.

% Though our approach is able to handle some refactorings, our approach
% cannot replace these approaches because (1) as discussed before, our
% approach do not consider multiple configurations, and (2) our approach
% does not change preprocessor directives, which is necessary in
% refactorings such as renaming macros and extract macros.

%Work by Christian Kastner
%
%Parsing C/C++ code without pre-processing by Y Padioleau, CC'09
%
%SuperC: parsing all of C by taming the preprocessor, PLDI'12

\subsection{C预处理器上的实例调查}
许多年来,学术界不乏各类对C预处理器十分有见解的学术实例调查工作
~\parencite{Spencer92,ernst2002empirical,Liebig2011}。
同时也有一些C预处理的替代品被提出,比如语法预处理器~\parencite{Weise1993,McCloskey:2005}
或者面向侧面开发方法~\parencite{Lohmann2006,Adams2009,Boucher2010}。
但是直至今日,业界并没有什么采用这类替代预处理器的迹象,
C预处理器依然是主流的工具。
这也说明C预处理器带来的麻烦依然广泛存在。



%%% Local Variables: 
%%% mode: latex
%%% TeX-master: "main"
%%% End: 
