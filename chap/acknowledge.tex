% vim:ts=4:sw=4
% Copyright (c) 2014 Casper Ti. Vector
% Public domain.

\chapter{致谢}

毕业论文写到这里告一段落,在此感谢大学四年来在学业上对我有过帮助的老师,亲友。

首先一定要感谢的是熊英飞老师。
熊英飞老师在我大二那年来到北大,我和他在计算机著名神课《计算机系统导论》上相识。
之后我加入熊老师的课题组,一直到今天,已经快2年半。
熊老师平日里一直是组里/所里的老好人,教授们喊他小熊,学生们喊他熊神。
他和学生完全打成一片,有教授的学识却无教授的架子。
也正是因为他和蔼,对学生容忍度很高,拖沓的我总是能钻空子,科研的时间虽长,但论文成果屈指可数。

然而在熊老师组内的日子并不是纯粹的游手好闲,每次去理科一号楼1431找熊老师时,总能看到他
或是在电脑前写论文、或是和同学讨论、或是对着白板思考问题。每当看到这一幕时,
我心中总有一种惭愧感。当然熊老师是不知道的,他总是会放下手里的事情接待我。
在这些找他的谈话中,作出了我大学生活中几个关键的决定:实习还是科研?选择哪所研究生院的录取?

我记得有一次闲聊时,听熊老师聊过当年他当年做学生时,有的老师完全不顾学生。
但当他遇到东京大学的胡振江教授后,他发现好的老师应该去感化学生,也因此立下志向做一名老师。
教学的理论我完全不了解,但作为学生,我大学四年真正地很幸运地被熊老师感化过。
我想,熊老师的确做到了他心中理想的教师的样子。
我为自己感到幸运,也为他感到高兴。

接下来感谢的名单不分先后。

感谢学校、学院对我的培养。大学四年中北大、信科给了我学术、工作上的舞台。我三次出国交流,
担任过学生干部,拿过“五四”最高荣誉等等。这中间离不开学校的栽培和包括梅宏院士、陆俊林老师、马郓学长等
学院老师前辈的鼓励和指导。

感谢黄震、赵明民、黄元三位室友。能和优秀的人相伴是我的荣幸。
你们三个总是霸占着年级前十和各种奖学金,赵明民还在本科期间就在MobiCom上发表文章。
这些光环总是鞭策着游业散漫的我前进。
前三年大家都太忙,感情尚可。但到了大四,四个人反而更坦诚相待,变得心心相惜起来。
兄弟这个词一般不适合我们四个从南方来的男生,
但我心里知道一生的友谊尽在不言中。
今年秋天我们四个天各一方,望时光不老,我们不散。

感谢我的爸爸妈妈,大学四年总没什么时间回家。谢谢你们对我一直的支持和关心。大四的暑假决定哪里也不去,留在家里陪爸妈。

感谢大学间认识的好友们,你们一直是我内心力量的来源。

感谢北京这座城市,接纳了一个讨厌又鄙视她的上海人四年。
感谢她提供了每一处地点都与我的一些故事相联系。
到头来,讨厌她的人也开始留恋。
我想这是包容。

至此致谢结束。我的大学生活即将画上句号。
希望未来的人生我能做对选择,
变成我想成为的那个自己。



