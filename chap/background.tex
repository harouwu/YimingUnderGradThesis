\chapter{背景介绍}
\label{sec:problem}



% \begin{itemize}
% \item what are C macros
% \item an example about why C macros are difficult
%   \begin{itemize}
%   \item needs to the choice for insertion
%   \item needs to show concatenation and single \#
%   \item needs to show high-order macros
%   \item needs to show \#if and other directives
%   \end{itemize}
% \item explain the complexity of C macros
% \item explain the complexity of propagating back
%   \begin{itemize}
%   \item the choice for insertion
%   \item dealing with concatenation
%   \end{itemize}
% \end{itemize}

% \begin{itemize}
% \item A table of C preprocessor directives
% \item An example to show the process of C preprocessing
% \item A change example (introduce a guarded condition) to discuss the bidirectional behavior
%   \begin{itemize}
%   \item show that changing macro definition is inappropriate
%   \item show that introducing new macro invocation is inappropriate
%   \end{itemize}
% \item More change examples
%   \begin{itemize}
%   \item change the variable name
%   \item break only the outer level
%   \item copying
%   \end{itemize}
% \item trivial solution 1: per file
%   \begin{itemize}
%   \item break a lot of macros
%   \item removing macro definitions
%   \end{itemize}
% \item trivial solution 2: per line
%   \begin{itemize}
%   \item still breaks useful macros( in the second example, or in
%     copying)
%   \item may produce erroneous result when a macro invocation spans
%     multiple lines
%   \end{itemize}
% \end{itemize}

\section{C预处理器} \label{sec:CPreprocessor}

\begin{table*}[htbp]
  \small
  %\centering
  \caption{主流预处理器指令与操作}
  \label{tab:preprocessor}
  \begin{tabular}{|p{3cm}|p{3cm}|l|p{4cm}|}
    \hline
    Directives & Functionality & Example & Result\\
    \hline
    \hline
    \code{\#progma} & Compiler options & \begin{lstlisting}
#progma once
\end{lstlisting} & removed from the preprocessed file\\
    \hline
    \code{\#include} & File Inclusion & \begin{lstlisting}
#include <stdio.h>
\end{lstlisting} & the content of "stdio.h"\\
    \hline
    \code{\#if, \#ifdef, \ldots} & Conditional compilation & \begin{lstlisting}
#ifdef FEATURE1
  x = x + 1;
#endif
\end{lstlisting} & \code{x = x + 1;}\\
    \hline
    \code{\#define X} & Object-like macro definition & \begin{lstlisting}
#define X 100
a = X;
\end{lstlisting} & \code{a = 100;}\\
    \hline
    \code{\#define X(a, b)} & Function-like macro definition & \begin{lstlisting}
#define F(x) x * 100
F(10);
\end{lstlisting} & \code{10 * 100;}\\
    \hline
    \code{a \#\# b} & Concatenation & \begin{lstlisting}
#define X a_##100
X
\end{lstlisting} & \code{a\_100}\\
    \hline
    \code{\#b} & Stringification & \begin{lstlisting}
#define F(x) #x;
F(hello);
\end{lstlisting} & \code{"hello";}\\
\hline
\code{\_\_FILE\_\_}, \code{\_\_DATE\_\_}, \ldots & Predefined macros &
                                                                       \code{\_\_FILE\_\_}
                                         & main.c \\
                                        
\hline
    
                                         
  \end{tabular}
\end{table*}

表~\ref{tab:preprocessor}显示了主流预处理器的指令与操作。
一条预处理指令在行首以\#开头,在行末结束。
宏可以被预处理指令定义,但是它们自身并不是预处理指令。
本质上,我们认为现在有四种主要的预处理指令:
\code{\#progma}~提供了编译选项,\code{\#include}~描述了包含的头文件,
\code{\#if}~提供了条件编译选项,\code{\#define}~是宏定义指令。
另外,在一个宏定义中,我们可以使用类似\code{\#\#}~和\code{\#}~的指令来连接两个变量、或字符化一个变量。
最后,还存在一些预定义的宏,如~\code{\_\_FILE\_\_}~,会随着上下文的不同而被替换。

当C预处理器处理一个源文件的时候,它会依据以下的方法来转换源程序文件:
\begin{itemize}
\item 首先展开~\code{\#include}~和~\code{\#if}~指令,然后再重复扫描展开后的代码词序列
\item 对于每个宏调用,预处理器先处理参数的展开,然后再展开宏调用。
\item 对于含有 \# 和 \#\#的参数,预处理器并不会处理这类参数。相反,预处理器会把参数直接文本拷贝到展开项中。
\item 当一个宏调用被展开之后,这个被展开的程序词序列会被再次扫描。
      如果这时还有宏没有展开,预处理器会把宏调用展开。
\item 为了避免循环展开宏调用,如果一个宏定义已经在展开过程中被展开,那么它就不会被再次展开。
\item 如果在展开宏的过程中生成了新的预处理指令,该指令并不会被预处理器执行。
\end{itemize}

\begin{figure}[ht]
  \centering
\begin{lstlisting}
#define SAFE_FREE(x) if (x) vfree(x);
#define FREE(x) vfree(x);
#define RESIZE(array, new_size, postprocess) \
  g_resize_times++; \
  postprocess(array); \
  array = vmalloc(sizeof(int)*(new_size));
#define GARRAY(x) g_array##x;

RESIZE(GARRAY(2), 100, FREE);
\end{lstlisting}
  \caption{一个预处理的例子\label{fig:example} }
\end{figure}

这里举一个实例,让我们来考虑在图~\ref{fig:example}里的代码片段。
这个例子向我们展示了许多实际项目中的宏定义与宏调用。
这段代码中含有四段宏定义和一个宏调用。
前两个宏定义被包含在一个用户自定义的空间释放函数里。
第三个宏定义是为了用户自定义的内存空间管理和日志记录而重新调整数组的大小。
最后一个宏是为了定义一组特殊的全局变量。
当预处理器扫描这段代码时,第一个参数~$RESIZE$~将会被处理。
此时,~\code{GARRAY(2)}~会被展开成~\code{g\_array2}~。
尽管第三个参数~\code{FREE}~已经被定义成一个函数状的宏(\emph{function-like macro}),
但是并没有能够提供给~\code{FREE}~的参数,因此此时预处理器并不会把它当作一个宏调用来处理。
然后~\code{RESIZE}~的宏调用被展开,于是我们得到了以下的代码:
\begin{lstlisting}
g_resize_times++;
FREE(g_array2);
g_array2 = malloc(sizeof(int)*(100));
\end{lstlisting}
现在我们能看到在展开的宏中,我们已经给~\code{FREE}~提供了一个参数列表。
因此接着系统会展开~\code{FREE(g\_array2)}~,我们得到以下代码:
\begin{lstlisting}
g_resize_times++;
vfree(g_array2);
g_array2 = malloc(sizeof(int)*(100));
\end{lstlisting}
换句话说,\code{RESIZE}~实际上是一个高阶宏(\emph{high-order macro}),
因为他的第三个参数也是宏。

\section{反向变换的设计}\label{sec:backdesign}
现在让我们来考虑一下输入为预处理后代码的程序编辑工具们。
比方在下面的例子中,程序编辑工具会发现~\code{vfree}~可能存在内存泄漏的可能,
所以工具会在这句代码前加入一个保护语句,如下:
\begin{lstlisting}
g_resize_times++;
if (g_array2) vfree(g_array2);
g_array2 = malloc(sizeof(int)*(100));
\end{lstlisting}

接下来的任务是生成反向变换。反向变换一般来说输入是预处理后代码上的修改,
然后产生预处理前代码上的修改。
当生成的修改作用在预处理前的代码上后,新的代码在预处理后会得到与作用输入修改的预处理后代码相同的结构。
对于之前例子里面程序编辑工具做出的修改,我们有两种处理办法:
(1)我们可以修改~\code{RESIZE}~的宏定义,把加入的~\code{IF}~语句加入到宏定义中;
(2)我们可以展开~\code{RESIZE}~的宏调用,并按照预处理后的修改把保护语句添加到原程序中。
第二个选项只影响到了这一小段局部的代码,而第一个选项有可能会影响到全局的其他宏调用。
但是,因为反向变换并不知道应该把修改作用到局部还是全局,一个更安全的做法是选择只影响局部的代码。
在本例中,有很大可能我们并不需要对每一个$vfree$的调用都加上保护语句,因为这样会带来大量不必要运行时间。
我们有理由相信程序编辑程序会根据需要选择是否在$vfree$前加上保护语句。
进一步讲,局部的选项会尽可能小地修改原有代码,因为全局会影响程序的许多部分。
这些可能性让我们想到一个双向预处理器的第一个性质。

\begin{decision}
反向变换不应该改变任何宏定义。
\end{decision}

根据我们定义的第一个性质,我们应该展开宏调用后再在源代码中加入保护语句。
然而现在在映射修改时我们又面临着以下两个展开的选择。我们可以把所有的宏都依次展开:

\begin{equation}
\begin{minipage}{7.6cm}
\begin{lstlisting}
g_resize_times++;
if (g_array2) vfree(g_array2);
g_array2 = malloc(sizeof(int)*(100));
\end{lstlisting}
\end{minipage}
\label{eqn:expandall}
\end{equation}
或者我们也可以只把宏展开一层:
\begin{equation}
\begin{minipage}{7.6cm}
\begin{lstlisting}
g_resize_times++;
if (GARRAY(2)) FREE(GARRAY(2));
GARRAY(2) = malloc(sizeof(int)*(100));
\end{lstlisting}
\end{minipage}
\label{eqn:goal}
\end{equation}
\newcommand{\coderef}[1]{code piece~(\ref{#1})}
我们认为代码段~\ref{eqn:goal}比代码段~\ref{eqn:expandall}更好,因为它保存了更多原有的结构,
这使得代码更加易懂,重用或是维护。
这就引入了我们的第二条性质。

\begin{decision}
反向变换应该尽可能保存现有的宏调用。
\end{decision}

也许有人会提议我们进一步地把这个保护语句缩减成一个新的宏,
或者复用现有的某一个宏来实现保护语句的功能。
在本例子中,我们的保护语句被~\code{SAFE\_MACRO}~这个新宏定义包含。
代码可能如下:
\begin{lstlisting}
g_resize_times++;
SAFE_FREE(GARRAY(2));
GARRAY(2) = malloc(sizeof(int)*(100));
\end{lstlisting}
但是,这样做是十分危险的。因为宏定义并不是用来替换所有语义相同的代码片段。
比方说,Ernst等人~\parencite{ernst2002empirical}就在文章中描述过某一个宏定义,
\lstinline!#define ISFUNC 0!, 定义了一个在系统调用中时常用到的常数。
很明显,我们并不能把整个系统里的~\code{0}都替换成~\code{ISFUNC}。
这就引入了我们的第三个性质。

\begin{decision}
  反向变换不应该引入新的宏调用。
\end{decision}

除了已经提到的三个性质之外,我们还有两条从双向变换领域借鉴过来的性质,
叫做 GETPUT 和 PUTGET~\parencite{Foster:2007}。
令$s$是预处理前的源程序,$t$是预处理后代码,$c_t$是作用在$t$上的变化,
$c_s$是完全的反向变换所提供的作用在$s$上的变化。

\begin{decision}[GETPUT]
  如果$c_t$是空的,那么$c_s$也是空的。
\end{decision}

\begin{decision}[PUTGET]
  令$s'=c_s(s)$为新的预处理前的源代码。意为把生成的$c_s$作用在$s$上。
  令$t'=c_t(t)$为作用了修改操作的预处理后代码。
  对$s'$做预处理会得到$t'$。
\end{decision}

% \begin{decision}
%   The backward transformation fails to produce $c_t$ if and only if
%   there exists no $c_t$ satisfying the above five requirements.
% \end{decision}

以上五条性质一起定义了反向变换的行为:它应该通过
修改宏调用参数、修改普通代码、展开宏调用并且展开尽可能少的宏调用
的方法来把预处理后的修改映射到源代码上。其中性质4与性质5又被认为是双向变换的正确性定义。~\parencite{Foster:2007}

\section{简单的方法}\label{sec:naive}
我们将在本节中讨论为什么简单的方法并不能满足我们提出的五条性质。\\

\noindent\emph{简单的方法 I (per-file).}
第一种简单的方法是直接把修改过的文件不加处理地拷贝覆盖源代码。
这种方法十分容易实现,我们称之为\emph{per-file},但是他有两大缺点。
首先,原有的未预处理源代码可能含有宏定义,然而现在的做法会让这些宏定义都丢失掉,
那么在其他地方,比如包含了该文件并调用了相应宏的其他代码,可能就会出现错误。
其次,这会导致把文件中所有的宏调用都展开,甚至包括所有的~\code{\#include}~指令。
整个代码面目全非,破坏了完整性。

\noindent\emph{简单的办法 II (per-line).}
第二个简单的办法是利用代码的性质,只把预处理后代码中被修改的行反向映射回源代码。
我们把这个方法称作\emph{per-line}。这个做法看起来是可行的,因为现代的预处理器
会记录下预处理前后行间的对应关系。
比方说,当GCC预处理我们之前的例子时,它会把1-7行替换成空行。
同时它会把从第9行开始的几行展开压缩成一行。
可以看到,现代预处理器都记录下了源代码与预处理代码行间的一一映射。

虽然第二种方法相较于第一种有了不删除宏定义的好处,它依然存在隐患。
首先,依然有不少宏被不必要地展开了。
在我们的例子中,如果我们把修改的那一行复制回去,那么代码段~\ref{eqn:expandall}就会时结果,
但这与我们定义的第二条性质不符。
甚至,如果我们考虑有的工具会在代码中挑选代码复制插入来修改,比如GenProg~\parencite{le2012genprog}
就会在代码中拷贝不同地方的代码来修改程序的错误。
这样一来拷贝的行中所有的宏调用都会被展开,代码的完整性还是被破环。
另外,这还并不是最严重的。
如果源文件中的宏调用是一个多行宏(在我们的调研与实验中也确实发现了这样的情况\secref{sec:evaluation}),
那么只替换宏调用的第一行会直接带来错误的结果,反而引入了新的bug。
比方说,如果在源代码的宏调用中插入了一个断行,如下:
\begin{lstlisting}
RESIZE(GARRAY(2),
  100, FREE);
\end{lstlisting}
在GCC中,这个宏调用会被展开成两行。其中第一行时全部的宏展开语句,而第二行是空行。
因此,修改操作只会作用于第一行。
如果反向变换仅仅把第一行替换成新的代码,那么就会留下不正确的程序。



%%% Local Variables: 
%%% mode: latex
%%% TeX-master: "main"
%%% End: 
