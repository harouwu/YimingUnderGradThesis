\chapter{算法设计}
\section{Approach}
\label{sec:approach}
正如之前提到的,我们算法的基本思想是把C预处理器当作一组重写规则,
而反向变换就是这些规则相应的逆向规则。
在本章钟,我们会先描述本项目的C预处理器的模型(\secref{sec:forward})。
然后我们会描述该系统所支持的修改操作。
接着我们会集中阐述其中第一种操作:替换操作的处理方法(\secref{sec:changes})。
因为我们需要吧每一条重写规则反向应用,因此我们需要记录下预处理时使用
重写规则的顺序(\secref{sec:changes}),然后按照顺序依次为这些规则生成反向变换(\secref{sec:steps})。
我们也会讨论这些步骤/过程为何能满足我们之前讨论的双向预处理器的五条性质(\secref{sec:correctness}),
并且给出一个更优化的算法。
最终,我们会讨论如何把不同类型的修改操作都转换成替换操作(\secref{sec:extend-other-changes})。


% We introduce our
% approach in the following steps. First, we present our model of
% forward preprocessing, which forms the basis of the correctness
% discussion (\secref{sec:forward}). Second, we introduce a model for
% encoding changes on the preprocessed code (\secref{sec:changes}) and
% describe how to backward transform one type of changes: replacement of
% a token
% (\secref{sec:backward}). % We introduce two core concepts, rewriting step and
% % rewriting action, for recording the process of a forward
% % transformation and perform the backward transformation.
% The backward transformation is based on tracing the forward
% transformation as rewriting steps (\secref{sec:steps}).
% We also
% discuss how this process satisfies the requirements and laws (\secref{sec:correctness}). Finally, we
% discuss how to convert other types of changes into replacement (\secref{sec:extend-other-changes}).

为了读者可以更快理解我们算法设计的思想,我们暂时只考虑C预处理器指令的一个子集:
去除 \# 操作和 \#\#操作,去除~\code{\#include}~操作,也不考虑宏出现循环调用的情况。
我们会在之后的章节中(\secref{sec:fullC})讨论如何把子集上的模型扩充到
支持全部C预处理指令的完整模型。

\newcommand{\dstart}{\ensuremath{\langle\#}\xspace}
\newcommand{\dend}{\ensuremath{\rangle}\xspace}
% \newcommand{\env}{\ensuremath{CTX}\xspace}

\subsection{模拟正向变换:预处理}\label{sec:forward}
我们把C预处理程序需要处理的程序看作词(\emph{token})的一个序列。
为了从词序列中识别出预处理指令,我们依赖于两个特征词:
行首的\#符号和该行最后的换行符。
我们同时也假设当前环境中所有定义的宏都存储在环境变量\emph{$context$}里。

我们把C预处理器语法看作是重写规则的一个集合。
每个重写规则都有$guard \hookrightarrow action$这样的形式。
当$guard$是真时,$action$会把当前词序列的前几项替换成指定的词,
然后在替换的位置后继续下一条替换指令。

在模型中,$guard$和$action$ 都可以被看作是函数。
$guard$函数输入输入时当先剩下的词序列和当前的上下文环境$context$,
它将会输出一个表示是否要把当前规则应用到现在的词序列的布尔值。
$action$函数把当前还剩下的词序列,上下文环境当作输入,然后生成
一个四元组 $(finalized, changed, 
restIndex, newContext)$。
这个四元组中的变量含义如下:
$restIndex$表示在这变量之前的词都已经被重写规则处理过了。
在这些被处理过的词序列中,$finalized$表示了一串不需要再应用规则的词序列,
而$changed$是可能还需要呗扫描的序列。
而最后一个变量$newContext$ 代表着更新过的上下文环境。

有了这些定义后,我们算法中的正向变换部分在算法~\ref{alg:forward}中展示。
该算法循环应用规则$R$直至整个词序列都被处理。


% , where $guard$ is a condition to
% determine whether this rule can be applied to the head of the current token
% sequence, and $action$ takes a subsequence from the beginning of the
% sequence, replaces it with another sequence, and separates the
% replaced sequence into 
% into a finalized sequence and the remaining sequence, where the
% finalized sequence do not need to be further scanned and the remaining
% sequence needs to be scanned. A forward transformation iterates
% all rules, and applies the first applicable rule until the no more
% token need to be scanned.

\begin{algorithm}
  \newcommand\mycommfont[1]{\rmfamily{#1}}
  \SetCommentSty{mycommfont}
  \caption{Algorithm for forward preprocessing  \label{alg:forward}}
  \KwIn{token sequence $src$, rule list $R$}
  \KwOut{new token sequence $res$}
  $ctx \leftarrow \{\}$\;
  \While{$src.length > 0$}{
    \For{$r \in R$}{
      \If{$r.guard(src, ctx)$} {
        $(finalized, hanged, rest, ctx') \leftarrow r.action(src, env)$\;
        break\;
      }
    }
    $res \leftarrow res + finalized$\;
    $src \leftarrow changed + src.sub(rest)$\;
    \tcp{$sub(l)$ returns a
      subsequence starting from $l$} 
    $ctx \leftarrow ctx'$\;
  }
 % {\small Function $guard$ takes the remaining token sequence and the current definitions as
 % input, and returns a Boolean value to denote whether the rule can be
 % applied. $action$ takes the token sequence and the definitions
 % as input, and returns a token sequence that needs not be scanned, a
 % token sequence needs to be further scanned, and an updated set of definitions.}
\end{algorithm}

在C预处理情况中,规则列表$R$中总共有四个规则。
其中一个处理条件预处理指令,例如$\#if$, $\#ifdef$;
一个处理其他的预处理指令,这样我们可以用它来清除预处理指令并且更新上下文环境;
一个处理宏调用;
最后一个处理普通字符文本。
这四个规则如下定义:

\begin{itemize}
\item 规则1: 这个规则处理条件编译选项。$guard$函数将判定当前词序列的开头是否是一个独立
  的预处理条件指令,例如$\#if$,$\#ifdef$。如果为真,
  $action$函数首先会在当前上下文环境中检验条件选项是否为真,然后根据是否为真选择使用
  真值分支或假值分支编译。
  选择分支后,用该分支替换原有的指令,并把替换成功的新词序列记录在$changed$里。
  而$finalized$是空的。
\item 规则2: 这个规则处理其余所有预处理指令。$guard$函数将判定当前词序列的开头是否是一个
  独立的预处理指令。如果为真,
  $action$函数会解析这条指令,并对当前上下文环境做出必要的调整,比如添加一条宏定义。
  最后,该指令之后的词下标将会被记录成$restIndex$,而$finalized$和$changed$为空。
\item 规则3: 这条规则会展开宏调用。$guard$函数将判定当前词序列的开头第一个词是否是一个
  对象宏调用(\emph{object-like macro})或者当前词序列的开头前两个非空白词是否是是一个
  函数宏调用(\emph{function-like macro})和一个开括号。如果为真,
  $action$函数会做以下两个步骤的操作:
  \begin{itemize}
  \item 首先,我们会用规则3和规则4循环调用处理参数\footnote{在C预处理器中未定义
    把预编译指令放在参数中的操作~\parencite{CStandard}。在次我们认为在
    参数中不存在预处理指令}。
  \item 然后,我们把宏展开中各个参数出现的位置都替换成已经处理完了的宏参数。
    整个展开的部分都被记录在$changed$里,而$finalized$是空的,$restIndex$记录
    了宏调用之后的第一个词的位置。
  \end{itemize}
\item 规则4: 这个规则处理那些没有被其余规则处理的普通文本自负。$guard$函数总是返回真。
  $action$函数会把词序列中的第一个词放入$finalized$中,返回一个空的$changed$,
  并把下一个词的下标标记为$restIndex$。
\end{itemize}

让我们来看一个例子来理解这些规则的应用。考虑以下程序。
\begin{equation}
\begin{minipage}{0.4\columnwidth}
\begin{lstlisting}
#define x 100
hello x
\end{lstlisting}
\end{minipage}
\label{eqn:smallexample}
\end{equation}
% To preprocess this program, we recursively apply the rules over the
% program. 
当我们的系统处理这段代码时,第一个能应用的规则是规则2。
它会把第一个宏定义解析出来,存储在上下文中,并把接下来的下标移动到下一行。
这时剩余的词序列为~\code{hello x}。
接着应用规则4,并把~\code{hello}移动到$finalized$的序列中。
然后应用规则3,把宏调用 \code{x} 展开成 \code{100}。
最终,系统将会再次扫描一遍 \code{100} 并使用规则4把它移动到$finalized$序列中。
当剩余词序列为空时,整个处理过程就停下了。

% From the definitions of the rules we can prove a simple property. For
% any rule $r$, let $r.guard(src, c)$ be true and $r.action(src, c) = (f, h, restIndex, c')$. Supposing
% $src'$ differs from $src$ only for the tokens at or after $restIndex$,
% we know that $r.guard(src, c)=true$ and $(f, h, restIndex, c') = r.guard(src', c)$. In other
% words, if we change the 

\subsection{模拟修改}\label{sec:changes}
程序编辑工具可以通过各种各样的方式对一段程序进行修改。
本文中我们考虑三种基本的修改:替换、插入和福祉。
这三种基本的修改是我们在分析了现有的主流代码修复工具
例如GenProg~\cite{le2012genprog} 和 SemFix~\cite{nguyen2013semfix}
之后总结而成。
这些代码修复工具往往会拷贝或创造一个语句来替换现有语句或插入来修改程序错误。

这三种操作都直接地对词序列进行操作。
一个替换操作描述为一个二元对$(l, s)$,其中$l$是要被替换的词的位置,
而$s$是将要替换在$l$位置的一串词序列。
一个插入操作也是一个二元对$(l, s)$,其中词序列$s$将会被插入到位置$l$
之后。
一个拷贝操作时一个三元组$(l, l_b, l_e)$。它表示从位置$l_b$(包含)起至
$l_e$(不包含)的词序列将会被拷贝插入到位置$l$的词之后。

我们可以看出替换操作涵盖的删除操作。一个形式为$(l, [])$
的替换操作就代表删除了一个词,其中$[]$表示空序列。

本章接下来的部分,我们将讨论如何实现一个可以处理替换操作的反向变换操作。
我们也会讨论如何把其他两种操作转换成替换操作。
% If a change set contains only
% replacements, we could also interpret the change set as a total function
% mapping each index into a token sequence, where unchanged indexes
% are mapped to the original tokens at the indexes.

\subsection{重写步骤}\label{sec:steps}
正如在序言中介绍过的,我们的反向变换操作会逆向实施之前提到的重写规则。
为了生成好的反向变换,我们设计了一个数据结构来记录下正向展开时
使用了哪些规则,分别应用在代码的哪个部分。
我们把这种记录的数据结构称作 \emph{重写步骤}。
每一个重写步骤都是一个四元组, $(src, i, ctx, r)$。
其中$r$表示规则$r$已经被应用到一段子序列$src$。
自序列位置在$i$,应用规则时的上下文环境为$ctx$。
其中$src$总是代表了暂时的词序列,区别于源序列与最终系列,是应用了
之前所有规则后生成的临时序列。

相应的,一个完整的正向序列会被记录成一个重写步骤序列。
比方说,正向处理之前例子中的代码段 \ref{eqn:smallexample} 时,
我们的算法会产生以下的重写步骤序列:
$(P, 0, \{\}, R2)$,$(hello\ x, 0, \{x\}, R4)$,$(hello\ x, 1, \{x\}, R3)$,和
$(hello\ 100, 1, \{x\}, R4)$。
其中$p$是源程序。

% We use a data structure to record how a rule application changes the
% token sequence, which is the key to our bidirectionalization. We call
% this data structure \emph{rewriting step}. Intuitively, we may first
% consider 
% A rewriting step is a
% triple, $(ctx, args, body)$. The first component, $ctx$, is the
% current context. The rest two steps contains two types of rewriting
% actions, argument expansions and body expansions, which are inner
% changes performed to form a rewriting step. Component $args$ contains
% a set of argument expansions, and $body$ is a body expansion.

% An argument expansion records how an argument to a macro invocation is
% expanded, and is recursively defined as a sequence of
% rewriting steps.

% A body expansion 


\subsection{考虑替换操作的反向变换}\label{sec:backward}
反向操作的基本思想是把预编译后代码上的修改通过重写规则序列
一步步映射到预处理前的代码上。这样当我们再跑一次正向变换时,
只会有两种情况:1. 程序会按照相同的重写规则再次正向展开
2. 程序会按照登记啊的重写规则序列展开,这个规则序列长度可以是0。
我们设计的反向变换操作在算法~\ref{alg:backward}中。
其中$backward$函数沿着每一个重写步骤映射程序上的修改,
或者是当它发现没有合适的修改可以映射时报错。

\begin{algorithm}
  \newcommand\mycommfont[1]{\rmfamily{#1}}
  \SetCommentSty{mycommfont}
  \caption{Algorithm for backward transformation \label{alg:backward}}
  \KwIn{a sequence of rewriting step $rs$}
  \KwIn{a set of replacements $c$}
  Reverse sequence $r$\;
  \For{$r \in rs$}{
    $c \leftarrow backward(rs, c)$\;
    \If{$backward$ failed}{print ``Changes cannot be applied.''\; \Return{}\;}
  }
  \Return{$c$}\;
\end{algorithm}

我们从伪代码中可以看出,实现反向变换的关键在于如何实现 $backward$ 函数。
$backward$ 函数的行为随着规则的不同而改变。
在此,我们根据复杂度,从简单到复杂讨论该函数的功能。

首先,如果当前重写步骤是依据规则2处理的,这意味着此处处理了一个非条件的
预编译指令,正向预编译时,这条指令被预处理器删除。反向变换会根据预处理
指令的位置来考虑是否要偏移操作。比方说,考虑以下的代码程序:
\begin{lstlisting}
#undef hello
hello
\end{lstlisting}
现在如果我们有一个$(0, x)$的替换操作,那么在这个例子中,
$hello$会被替换成$x$。
$backward$函数会把该替换操作位移到$(4, x)$来保证当前
替换操作依然是作用在$hello$这个词上。

第二,如果当前重写步骤是依据规则1处理的,这意味着此处预处理器
处理一个条件预处理指令。预处理器当时把这一段替换成了条件指令的一个分支。
在反向变换中,我们把在分支上发生的修改操作映射到原来的条件指令中,
同时计算必要的偏移量等。
因为我们不会改变宏的定义,此处的条件预处理指令在再次预处理时,
还是会选择相同的分支进行替换。

第三,如果当前重写步骤是依据规则4处理的,这意味着此处预处理器
处理了一段非指令非宏调用的普通文本字符。正向预处理器应该把一个
词放到$finalized$里去了。
反向变换需要保证在修改过后的词序列上,依然会应用规则4。
举例来说,如果用户把$hello\ c$ 中的 $hello$ 替换成了 $a\ b$,
我们需要检查,从$a$开始,是否唯一的预处理办法就是应用两次规则4来替换当前序列。
在这个例子中,只有当规则4的$guard$函数能够在要么 $a\ b\ c$ 或者 $b\ c$上返回真才行。
这时候就会发生反向变换失败的情况。
比方说,如果程序编辑程序给出的修改时把$hello$修改成$x$,
其中$hello$并不是宏调用,但$x$是宏调用。那么此时应该报错。
另一个情况是程序编辑程序把$GARRAY\ hello$替换成了$GARRAY\ (hello)$。
此时$GARRAY$是一个含有一个参数函数式宏调用。
在这种情况下,之前$GARRAY$被当作文本使用规则4重写因为它之后并不是括号。
而在新的序列中,$GARRAY$变成了一个宏展开。此时很有可能造成程序错误。
以上两种情况$backward$函数都应该报错,自动程序并不能轻易地把修改映射到代码中去。

% ???????????

最后,如果当前重写步骤是依据规则3处理的,这意味着此处预处理器
处理了一个宏调用。这种情况是最复杂的。
规则3包含了两个步骤。我们首先试图沿着两个步骤逆向重写代码。
如果两个步骤中有一个失败了,那我们只能试着展开宏调用。

在第二个小步骤中,宏展开中各个参数出现的位置都已经用展开的参数替换了。
如果有必要,反向变换需要把作用在展开后的参数上的修改都映射到参数上去。
这时有两种情况会导致逆向重写失败,也就是说我们不得不展开宏调用:
(1)被修改的词并不是宏的参数出现的位置中的;
(2)同一个参数在多个地方出现,被多次修改,但不同地方修改的内容不一样。
比方说,考虑以下代码段:
\begin{equation}\label{eqn:expansion}
  \begin{minipage}{0.8\columnwidth}
\begin{lstlisting}
#define x 1
#define plus(a) a+a
plus(x);
\end{lstlisting}
  \end{minipage}
\end{equation}
如果我们把 $1+1$ 变换成 $1-1$ 或者 $1+2$,我们在反向时就得展开宏调用。

在第一个小步骤中,我们预处理了宏的参数。
反向变换应该循环地调用$backward$来先把参数上的修改通过反向的重写步骤应用到每个参数上。
最终,我们需要对整个代码段在映射修改时做一次安全检查:
如果被修改的参数在最后含有了一个逗号,并且这个逗号没有被单独的括号包裹住;
或者是存在括号不匹配的情况,我们还是必须展开宏调用。
这是因为一个顶层的逗号或者不匹配的括号会破环原有的参数列表结构,影响程序的正确性。

如果在上述两个反向步骤中,反向变换失败,我们不得不展开宏调用。
展开宏调用时,会产生作用在宏展开式上的相应修改操作。
举个例子,我们不妨假设在上面的代码例子~\ref{eqn:expansion}中$1+1$中的第二个$1$被替换成$2$。
此时,映射来的修改会把 $plus(x)$ 中的 $plus$ 替换成 $x+2$ 然后删除掉 $(x)$。


% To
% perform the macro expansion, we need to generate the expanded form to
% replace the macro invocation.

在展开宏调用的过程中,关键是如何构造替换序列,在刚才的例子里时$x+2$。
从正向变换我们可以看出,没有变过的预处理词序列时 $1+1$, 
而且这两个 $1$ 都来自参数 $x$。
所以我们把参数 $x$ 的重写步骤拷贝到两个 $1$ 的词上。
然后使用这些重写步骤来反向映射作用在这两个 $1$ 上的修改操作。
第一个 $1$ 并没有变化,所以原来的 $x$ 可以继续保留。
但第二个 $1$ 被替换成了 $2$,因此该宏调用需要被展开,$x$ 也需要被展开。
于是我们得到了最终的替换文本 $x+2$。  

%We first construct an unchanged expanded form, and then
% propagate back the changes along the unchanged expanded form. First,
% we replace the occurrences of parameters in the macro body by either
% the preprocessed argument or the unpreprocessed argument (both the original
% arguments without user changes). An occurrence of a parameter is
% replaced by a preprocessed argument if any of the following conditions is
% satisfied.
% \begin{itemize}
% \item The preprocessed argument starts with a left parenthesis.
% \item The preprocessed argument contains a top-level comma.
% \item The preprocessed argument contains any unclosed parenthesis.
% \item The preprocessed argument becomes different if we preprocess it again.
% \end{itemize}
% Otherwise, the occurrence is replaced by an unpreprocessed argument. In
% the previous example, $x$ does not satisfy any of the above condition,
% so $a$ in the body of $plus$ is replaced by the unpreprocessed
% argument, forming $x+x$.

% Next, 
% for any
% occurrence of a parameter replaced by an unpreprocessed argument, the
% rewriting steps of the argument is first copied at the new location to
% preprocess the argument, and then we copy

但是,我们不能总是把参数的重写步骤原封不动滴拷贝到每个参数在展开式中出现的地方。
基本上说,当我们拷贝一个参数的重写步骤时,我们其实是在假设每一个参数
出现的位置可以用参数的未预处理形式来替换。而且替换后每个地方参数的
重写规则与它单独展开是一致的。
比方说,当展开 $plus(x)$ 时,我们可以把两个 $a$ 出现的地方替换成 $x$。
此时生成的展开式 $x+x$ 中两个参数出现的地方的展开重写规则序列和 $x$是一样的。
然后就能得到 $1+1$。
但是,情况并不总是这么理想,我们可以考虑以下的代码例子。

\begin{equation}\label{eqn:expandToParentheses}
  \begin{minipage}{0.8\columnwidth}
\begin{lstlisting}
#define p (x)
#define pplus(x) plus x hello
pplus(p)  
\end{lstlisting}
    \end{minipage}
\end{equation}

加上之前代码段 \ref{eqn:expansion} 中定义的宏,这里的例子最终会
预处理为$1+1\ hello$。但是,如果我们需要展开$pplus(p)$,
我们并不能把它直接展开成$plus\ p\ hello$,因为这只会展开成
$plus\ (1)\ hello$。这是因为在这里使用 $p$ 而不是 $(x)$ 
破坏了原有的宏调用。

%???????????

因此,我们需要在拷贝重写步骤的时候做一次安全检查。我们当且仅当
以下条件都不满足时,才会拷贝重写步骤。

\begin{itemize}
\item 预处理参数起始字符是一个左括号。
\item 预处理参数在顶层包含一个逗号。
\item 预处理参数中存在括号不匹配情况。
\item 如果我们重新预处理该参数,结果会不同。
\end{itemize}
前三个条件对应着之前例子中类似的情况:
预处理中参数在展开式中被用于其他的宏调用,替换它会导致展开式中的宏调用出错,
给程序引入错误。
% In the previous example, when $1+1$ is replaced by $1+2$, we expand
% the macro invocation $plus(x)$. Since the argument $x$ expands to
% $100$, the three conditions are satisfied and the body is replaced as
% $x+x$. Next we use the rewriting step of expanding $x$ to propagate
% the changes on the two $1$s. The first $1$ is not changed, so the
% original $x$ is kept. The second $1$ is changed into $2$, so the
% original macro invocation $x$ is also expanded, and we get the final
% text $x+2$, which is used to form a replacement of $plus$.
% To see an example where we should use the expanded argument, let us
% consider the following code.
% \begin{lstlisting}
% #define p (x)
% #define pplus(x) plus x hello
% pplus(p)  
% \end{lstlisting}
% With the definitions in code piece \ref{eqn:expansion}, the above code
% expands to $1+1 hello$. However, if we need to expand $pplus(p)$, we cannot
% expand it into $plus p hello$, because it will only expand to $plus
% (1) hello$. This is because the use of $p$ instead $(x)$ breaks the
% original macro invocation. In fact, all the first three conditions listed
% above are designed to prevent this case: the expanded macro is used as
% part of another macro invocation and replacing it with the unexpanded
% form may break the macro invocation.
最后一个条件对应着以下情况。
\begin{lstlisting}
#define id(x) x
id(plus p)
\end{lstlisting}
这段代码展开后得到$1+1$,但是如果我们把 $id(plus\ p)$
展开成  $plus\ p$,另外一个宏展开就会被阻塞。
换句话说,在宏调用中,一个参数会被扫描两次。
但是当宏调用被展开时,一个参数只会被扫描一次。
我们需要确认这之间的区别不会影响变换的正确性。

\subsubsection{正确性}\label{sec:correctness}
根据之前一节对反向变换的描述,我们可以简单推导出我们的反向变换满足
性质1、3、4和5。性质2将会在实验部分呗证实。
该反向变换满足性质1因为我们从不把修改操作映射到宏定义重。
该反向变换满足性质3因为如果我们通过修改操作引入了一个新的宏调用,
我们需要把一个普通的文本词变成宏调用的一部分。但是这个文本词本来
是通过规则4被预处理的。而规则4中反向变换的安全检查会组织错误的发生。
该反向变换满足性质4因为我们智慧在不得不展开某个宏调用时才会引入
新的修改操作,而该处没有修改操作,我们不会展开宏调用。
该反向变换满足性质5可以这样推导。
为了正式地证明性质5,我们需要诱导性地证明每一个重写步骤
只能依然有修改是保持同样行为,或者它会被一串等价的重写步骤
序列替换(该序列长度可以为0)。
所幸,当我们拷贝参数的重写步骤到展开式中时,
拷贝后的重写步骤可能会和之后的重写步骤混合起来。
我们可以指出,由于这些重写步骤序列作用位置不同且是覆盖性的,调换重写
步骤序列的位置并不会对结果产生影响。

\subsubsection{优化算法}\label{sec:optimization}
在现有的算法中,我们需要在每一步重写步骤后把整个程序的状态都记录下来,
即使只是很小的一部分发生了改变。
在反向变换中,每一步我们都需要调整所有修改操作的位移。
这样的算法使我们的系统十分笨重。

为了优化该算法,我们首先把源程序中的词序列切分成一组序列,
序列有自己独立的一套重写步骤。
这样,我们就能把每个序列当作单独的程序来做反向变换,再把修改融合到一起。
给定一个重写步骤 $(src, i, ctx, r)$,如果在 $i$ 位置的词不是由
之前的重写规则生成的,也就是说在原来的代码中就有,那么我们就可以认为
位置 $i$ 可以是一个\emph{切分点}。
有了切分点的定义后, 由于词是程序的最小单位,我们可以认为没有
重写规则会同时作用在切分点的两边。
我们依据切分点把程序切分成序列的集合,然后依次在序列上做反向变换,再融合这些修改。
当然,再合并两个子序列时,我们还需要特殊检查两边程序在预处理后,
合并起来不会形成新的宏调用。这个检查方法与之前提到的规则4的安全检查方法类似。

比方说,之前的代码段~\ref{eqn:smallexample}可以被切分成三个子序列:
宏定义, $hello$,和 $x$。
假设之前定义过宏 $plus$。 如果程序编辑工具把 $hello$ 修改成了 $plus$,
把 $100$ 修改成了 $(x)$, 那么两个子序列在预处理后,合并起来就在边界上
形成了一个新的宏调用。我们应该报错。


\subsection{扩展到完整的C预处理器}\label{sec:fullC}
本节中我们讨论如何把我们的算法扩展到支持完整的C预处理。
由于篇幅限制,我们只讨论核心思想而非细节。

为了在正向预编译中支持 \# 和 \#\#操作,我们需要在之前提到的规则3的
正向变换中添加两个新的小步骤。一个为了支持\#可以字符化一个词,一个支持
\#\#可以连接两个词。
另外,如果在参数中出现这两类操作,我们会直接使用未预处理的形式
来替换展开式中出现的位置。

在反向变换中,我们需要添加3个新的扩展。第一,我们需要为原有的规则3
依据之前所添加的小步骤,设计两个新的反向步骤来支持\#和\#\#操作。
第二,我们需要检查使用这两个操作的代码在预处理之前和预处理之后是否
有同样的修改。最后,当展开一个宏的时候,我们并不需要去回复\#和\#\#
操作,因为他们不能出现在没有宏定义指令的地方。

为了在正向变换中支持~\code{\#include}指令,我们需要为~\code{\#include}指令
添加新的一条重写规则。其内容为删除这行宏指令并引入新的文件内容。
在反向变换中,我们需要记录加入了哪些文件和内容,
并根据不同的~\code{\#include}指令,把这些文件引入都一一返回回去,
这样才能保持一致。

最后,一个完整的C预处理器不允许宏的循环调用。
为了支持这个功能,我们需要在正向重写规则中添加一些细节的考察。
当我们在不断重写由宏 $m$ 展开的词时,我们需要检查是否有再次使用这个宏展开的情况。
如果出现,应该报错。

\subsection{扩展到其他修改操作上}\label{sec:extend-other-changes}
在上文中我们详细地讨论了如何处理修改操作。
现在让我们来讨论如何处理插入和拷贝操作。
我们可以注意到,插入操作可以直接被转换为修改操作。
如果我们要在词 $x$ 之前插入一个词 $y$, 那我们可以把这个插入操作
转换为一个将 $x$ 变为  $y\ x$ 的替换操作。
但是,这样的转换有时会引入不必要的宏调用展开。
比方说,如果我门在之前的代码段~\ref{eqn:smallexample}中
在 $100$ 之前插入了一个词 $y$,那么我们在反向变换中并不需要拆开 $x$ 这个宏。
但如果我们把这个插入操作转换成一个替换操作的话,比如把 $100$ 变成 $y\ 100$,
那么我们就需要吧宏 $x$ 拆开。

刚才的例子是在切分点上插入了一段词,而这样的插入可以简单证明,不会影响到宏,
也不需要再反向编译时展开宏。
为了减少不必要的反向宏展开,我们把插入在切分点的修改操作
看作是添加了一个独立的子序列,然后直接在上面检查是否可以应用规则4。
如果其他规则可以在这个插入序列上应用,那说明这段插入是错误的,将会由我们的安全检查报错。
这样一来,其余的代码依然可以使用之前的算法处理。

拷贝操作的处理和插入操作类似。二者之间唯一的区别是,拷贝操作引入的代码段中
可能包含宏调用,而我们需要尽量不展这些宏调用。
为此,我们需要对这段词序列做一次反向转换。
这次反向转换算法如下:
(1)首先,我们假定对这段代码以外的所有预处理后代码做删除操作。
(2)然后,我们剔除掉那些没有在目标代码段定义的宏并做一次反向变换。
这样我们就能保证只有在目标代码段的上下文环境中定义的宏会被正确滴反向变换回去。
最后我们可以把已经做好反向变换的这段代码插入到预处理前的相应位置中。

% Instead, we deal with an insertion by concatenating three
% token sequences: the tokens before the insertion, the inserted tokens,
% and the tokens after the insertion. We first generate a change that
% deletes all tokens before the insertion, and perform a backward
% transformation to get the token sequence before insertion.
% In the above example, we first generates a change to delete $100$, and
% a backward transformation will give us 


% \subsection{Approach Overview}

% There are three key issues.
% \begin{itemize}
% \item We show that different preprocessor directives and macro
%   operators can be captured by a unified concept, text unit. The
%   execution of the preprocess can be captured by the composition of
%   text units, which enables a structural decomposition of the backward transformation.
% \item Even with the composition structure, we still need to deal with
%   the intersection of each type of text unit and each change
%   operation. We further show that there exists a core change
%   operation, replacement. We only need to consider the propagation of
%   replacement on each type of text unit. The other change operations
%   can be easily mapped to the core operation.
% \item Since our bidirectional transformation not only changes the
%   values transformed by the transformation but also the transformation
%   program itself, it will produce two outputs, one is the changed
%   program and one is the change to the data. When we apply the change
%   to the data to the program, we get the final result.
% \end{itemize}


% \newcommand{\replace}[2]{\ensuremath{[{#1}]\backslash {#2}}}
% \newcommand{\copyOpr}[3]{\ensuremath{[{#1},{#2}]\rightarrow [{#3}]}}
% \newcommand{\del}[2]{\ensuremath{\cancel{[{#1},{#2}]}}}
% \subsection{Modeling the Changes}

% A location or offset of a character in the source code file is 
% A location in the source code file is a pair $(f, o)$, where $f$ is
% the name of the source file, and $o$ is the offset from the start of
% the file, but treating all continuous whitespace characters as one character.

% A replacement is a pair $(l, s)$, where $l$ is a location on which the
% character is not a whitespace, and $s$ is
% a string that will replace the character at $l$.

% An insertion is a pair $(l, s)$, where $l$ is a location on which the
% character is a whitespace, and $s$ is a string that will be inserted
% at $l$.

% A copy is a pair $(l_t, l_b, l_e)$, showing the string between $l_b$
% and $l_e$ is copied into the location $l_t$, where $l$ is a location
% on which the character is a whitespace. 





% % Three atomic operations: replace, insert, copy, and delete.

% % Replace: a tuple $(p, s)$, denoted as $\replace{p}{s}$, showing the char at $p$ is replaced by $s$.

% % Copy: a tuple $(b, e, p)$, denoted as $\copyOpr{b}{e}{p}$, showing the string between $b$ and $e$ is
% % copied to replace the character at $p$.

% % Delete: a tuple $(b, e)$, denoted as $\del{b}{e}$, showing the string between $b$ and $e$ is
% % deleted.

% % Conflict rules: copy may overlap with the other two operations, but
% % the copied content is not modified. 

% % A set of standard operations: replace, copy, move, and delete.

% % RReplace: a tuple $(b, e, s)$, where the string between $b$ and $e$ is
% % replaced by $s$. RReplace can be modelled by Replace $(b, s)$ and
% % delete $(b, e+1)$.

% % CCopy is equal to copy.

% % DDelete is equal to delete.

% % MMove is a tuple $(b, e, p)$ that is equal to copy $(b, e, p)$ and
% % delete $(b, e)$.

% % \begin{definition}[Correctness]
  
% % \end{definition}


% % \begin{definition}[Completeness]
  
% % \end{definition}

% \subsection{Modeling Macro Expansion}
% % confluence in backward transformation? -- a sub case of correctness

% A text segment is a sequence of text units. A text unit is one of the
% following types: 




% \subsection{Forward Transformation}

% \subsection{Backward Transformation}

% We need to perform a check of the name collision

%%% Local Variables: 
%%% mode: latex
%%% TeX-master: "main"
%%% End: 
