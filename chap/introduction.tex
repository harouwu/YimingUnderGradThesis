% vim:ts=4:sw=4
% Copyright (c) 2014 Casper Ti. Vector
% Public domain.

\specialchap{序言}
% 中文测试文字。

\section{Introduction}
\label{sec:intro}

% \begin{itemize}
% \item program editing tools
% \item preprocessors
%   \begin{itemize}
%   \item used by mainstream languages, including OO language C++ and
%     ObjectiveC
%   \item casual uses
%   \end{itemize}
% \item program editing tools meet preprocessors
% \item reality: program editing tools do not support preprocessors
% \item existing work: C preprocessor in refactoring
%   \begin{itemize}
%   \item too complicated
%   \item not general solution
%   \end{itemize}
% \item contributions
% \item outline of the paper
% \end{itemize}

现在,许多程序分析工具都涉及代码修改功能。
在这些工具中,有许多都是代码修复工具~\parencite{le2012genprog,le2012systematic,QiMLDW14,nguyen2013semfix}。
通常来说,修复工具的输入是一段代码和一组测试,并不断修改代码直至代码能通过测试。
另一些程序分析工具是API升级工具~\parencite{li2015swin,Padioleau06,Meng:2011}。
当API升级时出现了不兼容情况时,这些工具可以自动更新相应的API调用让程序与API契合。
我们把这类直接修改代码的工具称作\emph{程序编辑工具}。

另一方面,许多程序语言的实现都带有预处理器~\parencite{ernst2002empirical,kohlbecker1986hygienic,lee2012marco}。
最常见的预处理器是C预处理器(C++)。
许多程序语言也接受C++,包括C,C++和Objective-C。
同时,程序员也时常使用C++来写一些零散的小工具。这时就会使用到预处理器。
例如Korpela~\parencite{Korpela2000}曾在文章中描述过用C++写一个HTML编辑工具:
这个工具会把页面间相同的HTML代码转换成C的宏,而不是直接生成HTML页面。
然后页面再利用这些宏最终生成HTML文件。

程序编辑工具通常不会去修改程序的预处理指令。
但是,只有能够把修改反映射到预处理之前的代码的工具才算有用。
只在预处理后的代码中修复错误会导致原有程序再次编译的时候错误依然存在,这样毫无意义。
这个问题具有挑战性,因为工具必须同时能理解预处理命令和目标程序语言,同时保证修改在两边能保持一致。
事实上,现有的程序编辑工具往往无法正确处理预处理指令、或是直接不处理预处理指令,例如现有的C语言工具:GenProg~\parencite{le2012genprog,le2012systematic},RSRepair~\parencite{QiMLDW14},和
SemFix~\parencite{nguyen2013semfix}。
这三个工具都只在预处理后的代码上工作。
用户需要手动检查预处理后的代码变化,并自行修改源代码——而这又增加了新bug的可能。

代码重构是一个密切相关的领域~\parencite{McCloskey:2005,Garrido2013}。
在代码重构中,程序编辑工具有时需要直接修改预处理指令。
比如:用户有时想重命名一个宏,或者需要提取一个宏作为重构的一部分。
在这种情况下,工具开发者别无选择,只好修改预处理指令。
典型情况中,工具开发者会定义一种新的C语法使得原有的C语法和预处理指令能兼容。
但是,如果我们考虑更一般的程序编辑工具,这种方法就捉襟见肘了。
首先,工具开发者需要在真正设计工具之前把精力花费在学习语言的细节上。
其次,学习新语言的努力并不能在其他语言中复用。

本文中我们提出了一个轻量级的支持C++的程序编辑工具实现方法。
该系统时一个双向的C++预处理器:原有的预处理过程可以背看作一个正向变换,在此基础上我们添加一个能够把预处理后代码上的修改反响映射回去的反向变换。
于是,程序编辑工具现在可以关注于预处理后的代码,并把映射修改的交给我们的自动工具\footnote{这个过程并不是全自动的。因为我们的工具现在支持程序编辑的基本操作。尽管这些步骤可以用通用代码差分方法实现~\parencite{fluri2007change},但是如果工具能直接提供基本编辑操作可以有最好的效果}。



%This idea is
%based on the observation that, unlike refactoring tools that often
%have to directly manipulate the preprocessor directives, many
%program-editing tools are not directly related to preprocessor
%directives and can be implemented more conveniently with only
%preprocessed programs.

在这里我们列举一些例子:
(1)上文中所提到的三个学术届认可的错误修复系统现在可以处理预处理前的代码;
(2)API升级软件现在可以在有预处理代码的情况下更好地实现;
(3)所有并不需要关心预处理过程的程序编辑工具都能够被改善。


虽然现在存在着几种双向变换的技术~\parencite{MaHNHT07,Voigtlander09bff,voigtlander2010combining},但是它们都是为数据的转换设计的。
给定一个源数据集$s$,一个变换程序$p$,和一个目标数据集$t=p(s)$,这些方法试图将$t$上的变化描述成$s$的变化。
然而,C的预处理器与数据的转换不同,因为C的源代码不仅包含了作为数据的代码,还包含了座位变换程序的预处理指令。
这就要求反向变换的机制能处理更复杂的情况:当目标数据发生变化时,我们可能要变化源数据、转换程序、或者是二者都变化。

%It is not easy to bidirectionalize CPP. 

本系统的一个创新处是在能够处理上文提到的复杂双向变换情况的同时,尽量最小化对变换程序$p$的修改。
首先,该设计中从不引入新的宏定义或修改现有的宏定义,有效地限制反向变化的影响。
其次,该设计只会移除/修改宏调用而并不会创造新的宏调用。
再有,该设计只会在必要的时候移除宏调用。这样,我们就能尽可能保证源代码和修改之前的相似度。


实现这样的算法设计也是十分具有挑战的。
典型的这类双向变换的方法~\parencite{MaHNHT07,Voigtlander09bff,MMHT10}是顺着程序的抽象语法树AST拆分双向变换。
每一个子树对应着一个小的双向变换,组合起来就成为整个程序的双向变换。
但是,一个未预处理的C++的程序并不能简单地解析成树状结构。
比方说,在下列代码中,
\begin{lstlisting}
#define inc(x) 1+x
#define double(x) 2*x
inc(double) 2
inc(double) (x)
\end{lstlisting}
第一个宏调用\code{inc(double)}会自动展开成为一个单独的$statement$。
但是第二个宏调用不能单独展开成一个$statement$。它需要连上之后的\code{(x)},循环展开后才算完整的句子。
因此,我们不能把第二个宏调用\code{inc(double)}当作独立的一段并直接对他做双向变换的分析。
为了克服这个困难,我们为描述类似C++程序这样的情况设计了新的模型。
我们把预处理看作是对代码数据的重写规则(\emph{rewriting rules})集,而不是直接把代码解析成抽象语法树。
这样的模型把程序的双向变换看作是重写规则的双向变换。


另外,该设计也可以被证明满足双向变换的正确性:
(1)如果预处理后的程序没有变化,那么源代码也不会变化;
(2)源代码在接受了反向映射的变化后,预处理后会得到相同的变化后的预处理后的代码。
这两个性质在双向变化领域被称作GETPUT和PUTGET~\parencite{Foster:2007}。

总之,该项目的贡献如下:
\begin{itemize}
% \item We propose a lightweight approach to support handling of the C
%   preprocessor in program-editing tools. Our approach bidirectionalizes
%   CPP, reflecting the changes on the preprocessed code
%   back to the unpreprocessed code.
% \item We demonstrate that our approach satisfies the two basic properties
%   of bidirectional transformation, GETPUT and PUTGET, in the context
%   of CPP.
\item 我们提出了一个轻量级的在程序编辑工具方面的双向C预处理器。
	我们分析了不同可能的设计并提出了五个反向变换应有的性质,
	包括GETPUT和PUTGET
	(\secref{sec:problem})。
\item 我们提出了一个能够符合五个性质的算法。
	该算法把C的预处理看作是重写规则的集合,
	并结构性地把预处理的双向变换转换到重写规则的双向变换上
	(\secref{sec:approach})。
\item 我们在Linux内核上验证了我们的方法,并和另外两种基本的
	做法进行比较:一个是直接把整个修改的文件映射回去;另一个是只把修改了的代码行映射回源代码。
	实验的结果显示相较于其他方法,我们的方法破坏了相当少的宏调用,
	并且总是可以给出正确的修改,而其他方法有时不行(\secref{sec:evaluation})。
\end{itemize}

本文接下的部分组织如下:
首先我们会在~\secref{sec:problem}中描述项目背景和需要解决的问题。
接着我们会在~\secref{sec:approach}中提出我们的算法,证明它满足双向预编译器的五种性质。
然后我们会在~\secref{sec:evaluation}中给出我们的实现与实验的过程、细节与结果,并进行讨论分析。
最后我们会在~\secref{sec:related}中讨论相关工作,并在~\secref{sec:conclusion}中总结。


% The rest of the paper is organized as follows. \todo{finish this}.

%Finally, we discuss related work in %\secref{sec:related} and conclude
%the paper in %\secref{sec:conclusion}.