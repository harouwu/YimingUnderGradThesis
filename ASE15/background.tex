\section{Problem}\label{sec:problem}



% \begin{itemize}
% \item what are C macros
% \item an example about why C macros are difficult
%   \begin{itemize}
%   \item needs to the choice for insertion
%   \item needs to show concatenation and single \#
%   \item needs to show high-order macros
%   \item needs to show \#if and other directives
%   \end{itemize}
% \item explain the complexity of C macros
% \item explain the complexity of propagating back
%   \begin{itemize}
%   \item the choice for insertion
%   \item dealing with concatenation
%   \end{itemize}
% \end{itemize}

% \begin{itemize}
% \item A table of C preprocessor directives
% \item An example to show the process of C preprocessing
% \item A change example (introduce a guarded condition) to discuss the bidirectional behavior
%   \begin{itemize}
%   \item show that changing macro definition is inappropriate
%   \item show that introducing new macro invocation is inappropriate
%   \end{itemize}
% \item More change examples
%   \begin{itemize}
%   \item change the variable name
%   \item break only the outer level
%   \item copying
%   \end{itemize}
% \item trivial solution 1: per file
%   \begin{itemize}
%   \item break a lot of macros
%   \item removing macro definitions
%   \end{itemize}
% \item trivial solution 2: per line
%   \begin{itemize}
%   \item still breaks useful macros( in the second example, or in
%     copying)
%   \item may produce erroneous result when a macro invocation spans
%     multiple lines
%   \end{itemize}
% \end{itemize}

\subsection{The C Preprocessor}

\begin{table*}[htbp]
  \centering
  \caption{Main preprocessor directives and operators}
  \label{tab:preprocessor}
  \begin{tabular}{|c|c|l|p{4cm}|}
    \hline
    Directives & Functionality & Example & Result\\
    \hline
    \hline
    \code{\#progma} & Compiler options & \begin{lstlisting}
#progma once
\end{lstlisting} & removed from the preprocessed file\\
    \hline
    \code{\#include} & File Inclusion & \begin{lstlisting}
#include <stdio.h>
\end{lstlisting} & the content of "stdio.h"\\
    \hline
    \code{\#if, \#ifdef, \ldots} & Conditional compilation & \begin{lstlisting}
#ifdef FEATURE1
  x = x + 1;
#endif
\end{lstlisting} & \code{x = x + 1;}\\
    \hline
    \code{\#define X} & Object-like macro definition & \begin{lstlisting}
#define X 100
a = X;
\end{lstlisting} & \code{a = 100;}\\
    \hline
    \code{\#define X(a, b)} & Function-like macro definition & \begin{lstlisting}
#define F(x) x * 100
F(10);
\end{lstlisting} & \code{10 * 100;}\\
    \hline
    \code{a \#\# b} & Concatenation & \begin{lstlisting}
#define X a_##100
X
\end{lstlisting} & \code{a\_100}\\
    \hline
    \code{\#b} & Stringification & \begin{lstlisting}
#define F(x) #x;
F(hello);
\end{lstlisting} & \code{"hello";}\\
\hline
\code{\_\_FILE\_\_}, \code{\_\_DATE\_\_}, \ldots & Predefined macros &
                                                                       \code{\_\_FILE\_\_}
                                         & main.c \\
                                        
\hline
    
                                         
  \end{tabular}
\end{table*}

Table~\ref{tab:preprocessor} shows the main preprocessor directives
and operators. A preprocessor directive starts with a \# at the
beginning of the line and ends at the end of the line. Macros can be
defined by preprocessor directives but themselves are not preprocessor directives.
Basically, we have four main types of preprocessor directives:
\code{\#progma} providing compilation options, \code{\#include} for including header files, \code{\#if} for
conditional compilation, and \code{\#define} for macro definitions. Additionally, within
a macro definition, we can use operators such as \code{\#\#} and \code{\#}, for
concatenating two variables or quoting a variable. Finally, there are
some pre-defined macros such as \code{\_\_FILE\_\_}, which will be replaced by
the current values of the fields. 

When the C preprocessor processes a source file, it transforms the source file according to the following rules:
\begin{itemize}
\item The \code{\#include} and \code{\#if} directives are
  first expanded, and then the expanded token sequences are scanned.
\item For each macro invocation, the arguments are first preprocessed, and then
  the invocation is expanded.
\item If an argument contains \# or \#\#, the unpreprocessed argument is
  used, otherwise the preprocessed argument is used.
\item After a macro invocation is expanded, the expanded token
  stream is scanned again, where any newly introduced macro
  invocations are again expanded.
\item To avoid infinite recursive expansion, if a macro has been
  expanded during the expansion process, it will not be expanded
  again.
\item Any new preprocessor directives produced in the expansions will
  not be used in processing. 
\end{itemize}


\begin{figure}[ht]
  \centering
\begin{lstlisting}
#define SAFE_FREE(x) if (x) vfree(x);
#define FREE(x) vfree(x);
#define RESIZE(array, new_size, postprocess) \
  g_resize_times++; \
  postprocess(array); \
  array = vmalloc(sizeof(int)*(new_size));
#define GARRAY(x) g_array##x;

RESIZE(GARRAY(2), 100, FREE);
\end{lstlisting}
  \caption{An example of code preprocessing\label{fig:example} }
\end{figure}

As a concrete example, let us consider the code in
\figref{fig:example}. This example is contrived to exhibit many real
world patterns of macro uses.  This piece of code contains four macro
definitions and a macro invocation. The first two macro definitions
are wrappers for a user-defined free function. The third one is for
resizing an array with user-defined memory management and logging. The last one is
for naming a set of special global variables.
When the code is scanned by the preprocessor, first the arguments of $RESIZE$ are processed. In
this case, \code{GARRAY(2)} is expanded into \code{g\_array2}. Though
the third parameter \code{FREE} has been defined as a function-like
macro, no argument is provided for \code{FREE}, so the preprocessor
does not treat it as a macro invocation at this stage. Then the macro
invocation to \code{RESIZE} is expanded, leading to the following
code:
\begin{lstlisting}
g_resize_times++;
FREE(g_array2);
g_array2 = malloc(sizeof(int)*(100));
\end{lstlisting}
Now we
can see that in the expanded macro, an argument list is actually
provided for \code{FREE}. Then the system will expand
\code{FREE(g\_array2)}, resulting in the following code. 
\begin{lstlisting}
g_resize_times++;
vfree(g_array2);
g_array2 = malloc(sizeof(int)*(100));
\end{lstlisting}
In
other words, \code{RESIZE} is in fact a high-order macro
where its third parameter is also a macro.

\subsection{Designing the Backward Transformation}
Now let us consider that a program-editing tool takes the preprocessed
code. Realizing the invocation to \code{vfree} is unsafe, the
programming editing tool inserts a guard before it, as
shown below.
\begin{lstlisting}
g_resize_times++;
if (g_array2) vfree(g_array2);
g_array2 = malloc(sizeof(int)*(100));
\end{lstlisting}


Now let us consider how to design a backward transformation that takes
this change as input, and produces a change on the unpreprocessed code,
where the changed unpreprocessed code will produce exactly the changed
preprocessed code. There are two options: 1. we can change the definition of
\code{RESIZE} macro, adding this guard into the macro definition.
2. we can expand the invocation to \code{RESIZE} macro, and add
the guard to the expanded form. The first option makes global changes,
affecting all places that invoke \code{RESIZE}. The second option
makes only local changes, affecting only one place. However, since
the backward transformation do not have the knowledge whether the change should be
global or local, it is safer to choose the local option. In this example,
probably we do not need to add the guard to every invocation to
$vfree$, as it will cause unnecessary runtime overhead. The
program-editing tool will have the knowledge to determine the safety
of $vfree$ calls and add this guard to all necessary places.
Furthermore,
the local option also minimizes the interruption to the original
program, as the global option affect many places in the program. This consideration forms our first requirement of the
backward transformation.

\begin{decision}
A backward transformation should not change any macro definition.
\end{decision}

According to the requirement, we should expand the macro
invocation with the guard added. Then there are again two
options. We may subsequently expand all macro expansions as follows.
\begin{equation}
\begin{minipage}{7.6cm}
\begin{lstlisting}
g_resize_times++;
if (g_array2) vfree(g_array2);
g_array2 = malloc(sizeof(int)*(100));
\end{lstlisting}
\end{minipage}
\label{eqn:expandall}
\end{equation}
Or we may expand only one level as follows.
\begin{equation}
\begin{minipage}{7.6cm}
\begin{lstlisting}
g_resize_times++;
if (GARRAY(2)) FREE(GARRAY(2));
GARRAY(2) = malloc(sizeof(int)*(100));
\end{lstlisting}
\end{minipage}
\label{eqn:goal}
\end{equation}
\newcommand{\coderef}[1]{code piece~(\ref{#1})}
We think \coderef{eqn:goal} is better than \coderef{eqn:expandall} because it
keeps more of the original code structure that are easier to understand, reuse, and maintain. This leads to our second
requirement.

\begin{decision}
A backward transformation should aim to keep existing macro invocations.
\end{decision}

Some might be tempted to go a step further by abstracting the guard into a new macro, or
trying to reuse an existing macro that covers its functionality, as in the following. 
%Furthermore, the added guard actually has been captured by another
%macro definition: \code{SAFE\_MACRO}, so it is attempting to replaced
%the second line with an invocation to existing macros, as follows.
\begin{lstlisting}
g_resize_times++;
SAFE_FREE(GARRAY(2));
GARRAY(2) = malloc(sizeof(int)*(100));
\end{lstlisting}
However, this is dangerous as very often macro definitions are
 not designed to replace all occurrences of its expanded
form. For example, Ernst et al.~\cite{ernst2002empirical} reports a
macro definition, \lstinline!#define ISFUNC 0!, that defines a constant used in certain system
calls. It obviously does not mean we shall replace all occurrences of \code{0}
with \code{ISFUNC}.
%but a backward transformation does not have the
%information to determine where an invocation should be inserted. 
This
leads to our third requirement.

\begin{decision}
  A backward transformation should not introduce new macro invocations.
\end{decision}

Besides the three requirements, we have two additional correctness laws borrowed from the bidirectional transformation literature, namely GETPUT and
PUTGET~\cite{Foster:2007}. Let $s$ be the unpreprocessed
program, $t$ be the preprocessed program, $c_t$ be the change to $t$,
and $c_s$ be the change to $s$ that is produced by a successful
backward transformation.

\begin{decision}[GETPUT]
  If $c_t$ is empty, then $c_s$ is empty.
\end{decision}

\begin{decision}[PUTGET]
  Let $s'=c_s(s)$ be the new unpreprocessed program obtained by
  applying $c_s$ to $s$, and $t'=c_t(t)$ be the changed preprocessed
  program. Preprocessing $s'$ produces $t'$.
\end{decision}

% \begin{decision}
%   The backward transformation fails to produce $c_t$ if and only if
%   there exists no $c_t$ satisfying the above five requirements.
% \end{decision}

The requirements together define the behavior of a backward
transformation: it should propagate the change
back into the unpreprocessed code by means of modifying macro parameters,
modifying normal text, and expanding macro invocations, and at the same time expanding as
fewer macro invocations as possible. 

\subsection{Naive Solutions}\label{sec:naive}
Naive solutions does not meet our above listed requirements as we demonstrate in this subsection. \\

\noindent\emph{Naive solution I (per-file).}
The first naive solution is to directly copy back the changed
preprocessed files
and replace the original unpreprocessed files. This solution is the easiest to
implement but has two major deficiencies. First, the original unpreprocessed source files may contain
macro definitions, which will be lost if we directly
copy back the preprocessed files, causing problems elsewhere. Second, this will leads to the expansion
of all macro invocations in the source file, as well as all \code{\#include}
directives, destroying modularity completely.\\
% A source file often contains thousands of lines expanded
%from \code{\#include} directives, and thus such a backward
%transformation often leads to almost unreadable source files.

\noindent\emph{Naive solution II (per-line).}
The second solution is to use more fine-grained units, copying back
only the changed lines. This is doable with the assumption that modern
compilers keep a traceability relationship between the lines
in the unpreprocessed files and the preprocessed files. For example,
when GCC preprocess our running example, it will replace lines 1-7 into
empty lines, and put the three statements expanded from line 9 in one
line, so that a one-to-one correspondence between lines are kept. 

Although the second solution has the benefit of not removing macro
definitions, it still has problems. First, a lot of
macros are still unnecessarily expanded. In our example, if we copy back
the changed line, \coderef{eqn:expandall} will be the result, which goes against our second 
requirement and destroys modularity. The situation becomes even worse when we consider tools that
copy lines of code within source files, e.g.,
GenProg~\cite{le2012genprog} copies statements between two positions
in the source files to fix bugs. In such cases all macro invocations
in the copied lines will be lost. Second, if the original macro
invocations span several lines, the backward mapping produces wrong
results. For example, suppose there is a line break within the
original macro invocation, as follows.
\begin{lstlisting}
RESIZE(GARRAY(2),
  100, FREE);
\end{lstlisting}
In GCC, the macro invocation will be expanded into two lines, where
the first line contains the three expanded statements, and the second
line is empty. As a result, in the backward transformation only the
first line will be copied back, resulting in an incorrect program.



%%% Local Variables: 
%%% mode: latex
%%% TeX-master: "main"
%%% End: 
