\section{Related Work}
\label{sec:related}

\subsection{Bidirectional Transformation}
Our work is inspired by research on bidirectional transformation. 
%The need to execute programs in both forward and backward directions is widespread. 
A classical scenario is the \emph{view-update problem}~\cite{BaSp81,DaBe82,Hegner90,Cui2000,Fegaras2010} from database design: a view represents a database computed for 
a source by a query, and the problem comes when translating an update of the view back to a corresponding update on the source. 
%The problem is central in the same way to \emph{model transformations}, playing a substantial role in software evolution:
%having transformed a high-level model into a lower-level implementation, for a variety of reasons one often needs to reverse engineer a
%revised higher-level model from an updated implementation~\cite{YuLHHKM12,XiLHZTM07}. 

Languages have been designed to streamline the development of such applications involving transformations running
bidirectionally. Notably the \emph{lenses} framework~\cite{HuMT04,MuHT04aplas,Foster:2007,BoFPPS08,FoPP08,WaGMH10,Diskin2011,Hofmann2012,FoMV12,RaLFC13}, covering a number of languages that provide bidirectional combinators as language constructs.
%, and bidirectional programs built through such combinators are correct by construction. 
A different approach 
%in designing bidirectional programs 
is to mechanically transform existing unidirectional programs to obtain a backward counterpart, a technique known as \emph{bidirectionalization}~\cite{MaHNHT07,Voigtlander09bff,voigtlander2010combining,WaGW11,VoHMW13,WaGMH13,MaWa13,MaWa13ppdp,WaNa14,MaWa14}. In the software model transformation literature, the underlying data to be transformed are usually in the form of graphs (instead of trees), and a relational (as oppose to functional) approach that specifies the bidirectional mappings between different 
model formats is more common~\cite{qvt,Stevens2010,Schurr1994,Schurr2008,HiHIKMN10,Hidaka:2011}. 
However, the requirement of our work goes beyond what these languages offer: in our framework, not only data, but also transformations (macros) are subject to bidirectional updates.
%The latter offers a more convenient programming style by not restricting programmers to a small set of stylised combinators; but since the backward transformations are mechanically generated, finer control of their behaviors is lacking. 

%A common trait of most bidirectional languages is the state-based updating: the state of the updated view, instead of 
%the operation performing the update, is used as the input of the backward transformation. Exceptions do exist; and additional information
%about how, on top of what, updates are made are used to improve efficiency~\cite{WaGW11} and precision~\cite{Diskin2011,Hofmann2012} of backward transformations. 


%It will be interesting to see whether our implementation of bidirectional preprocessor can be implemented in the above 
%languages. Such an exploration is orthogonal to our aim in this paper and is expected to be non-trivial. 


%\emph{QVT} (Query/View/Transformation)\emph{-Relations}~\cite{qvt,Stevens2010} is a declarative language standardised by the Object Management Group, permitting both unidirectional and bidirectional model transformations to be written. 
%But the standard falls short of a well-defined operational semantics.
%\emph{TGG} (Triple Graph Grammar)~\cite{Schurr1994,Schurr2008} is a competing framework for specifying traceability links between different types of models. Different from QVT, TGG is highly operational and updates of models are handled incrementally. A notable exception of the above mentioned relational approach is \emph{GroundTram}~\cite{HiHIKMN10,Hidaka:2011}, where unidirectional queries on models serve as specifications and are bidiretionalized by the system. {\bf Xiong's papers?} 

%The use of bidirectional transformation techniques in programming language design and implementation has become popular recently, 
%particularly aiming at improved user experience~\cite{WaGMH10,haskell-lens, WaGMH13,MaWa13}. Our effort in this paper adds to the list, where a bidirectional C preprocessor is key to support comprehensible feedback to programmers. 

\subsection{Analyzing and editing unpreprocessed C code}
The C preprocessor poses a great challenge for static program
analyses. The ability of producing a number of possible preprocessed
variants causes a combinatorial explosion, rendering it infeasible to
employee traditional tools that are designed to analyze a single
variant at a time. Only until very recently, sound parsing and
analyzing unpreprocessed C code is made possible through
\emph{family-based
  analyses}~\cite{Kastner2011,Gazzillo2012,Liebig2013}. Earlier tools
have to resort to unsound heuristics or restrict to specific usage
patterns~\cite{Baxter2001,Garrido2005,Padioleau2009}.

Similarly, a lot of efforts in refactoring C code are devoted into
dealing with multiple variants. Most approaches
\cite{Garrido2002,Vittek2003,Spinellis2003,Garrido2013} try to find a
suitable model that represent both the C program and the preprocessor
directives. A recent approach~\cite{Overbey2014} suggests an
alternative: perform refactoring on one variant and prevent the
refactoring if problems may be caused in other variants. This is based
on the observation that changes on one variant seldom causes problems
in other variant.

Unlike these approaches, our approach currently considers only one
variant. In the future we may combine our approach with these
approaches to deal with multiple variants. However, handling only one
variant is already useful in many cases: (1) many programs, though
with conditional compilation, do not have many variants; (2) as
revealed by Overbey et al.~\cite{Overbey2014}, changes in one variant
often do not cause problems in other variants.%  However, our approach requires minimal
% changes to existing tools, so can potentially be applied to many areas
% that do not require specialized effort.

% Refactoring unpreprocessed C code is extremely difficult. Even without considering the above mentioned problem with parsing, dealing with macro expansion while editing remains tricky~\cite{Garrido2002,Vittek2003,Spinellis2003,Garrido2013,Overbey2014}. Even for relatively simple tasks such as identifying and removing dead code sophisticated analyses are required~\cite{Baxter2001,Tartler2011}. 

% Our proposal in this paper does not apply to the cases here. We do
% not consider multiple configurations, and we do not allow explicit
% modification of preprocessor directives.

% Though our approach is able to handle some refactorings, our approach
% cannot replace these approaches because (1) as discussed before, our
% approach do not consider multiple configurations, and (2) our approach
% does not change preprocessor directives, which is necessary in
% refactorings such as renaming macros and extract macros.

%Work by Christian Kastner
%
%Parsing C/C++ code without pre-processing by Y Padioleau, CC'09
%
%SuperC: parsing all of C by taming the preprocessor, PLDI'12

\subsection{Empirical studies on the C preprocessors}
Over the years, there has been no shortage of academic empirical
studies that are critical towards the C
preprocessor~\cite{Spencer92,ernst2002empirical,Liebig2011}, and
replacements of CPP are proposed such as syntactical preprocessors~\cite{Weise1993,McCloskey:2005} and aspect-oriented programming~\cite{Lohmann2006,Adams2009,Boucher2010} are plenty. However until present, there is no sign of any adoption of these alternatives in industry, with the C preprocessor is still being seen as the tool of the trade~\cite{Medeiros2015}. 


%%% Local Variables: 
%%% mode: latex
%%% TeX-master: "main"
%%% End: 
